\documentclass{article}
\usepackage[margin=1in]{geometry}
\usepackage{amsmath, amsthm, amssymb, amsfonts}
\usepackage{graphicx}
\usepackage{xcolor}
\usepackage{enumerate}
\usepackage{environ}

\newcommand{\inv}{^{-1}}

\newcounter{pnum}
\NewEnviron{problem}{
    \stepcounter{pnum}
    \begin{center}
        \fbox{
        \begin{minipage}{0.95\linewidth}
            \textbf{\thepnum.} \BODY
        \end{minipage}}
    \end{center}

    \textbf{Solution. }
}

\title{\vspace{-4em}Homework 7 Bonus}
\author{Hanzhang Yin}
\begin{document}

\maketitle

\begin{problem}
    [Bonus]
        Let \(A\) and \(B\) be \(n\times n\) matrices.
        \begin{enumerate}[a)]
            \item Prove that
            \[e^{A+B} = e^Ae^B\]
            when \(AB=BA\).
            \item Give an example where
            \[e^{A+B} \neq e^Ae^B.\]
        \end{enumerate}
    \end{problem}
    
    \subsubsection*{(a)}
    \begin{proof}
        The matrix exponential $e^M$ of a square matrix $M$ is defined by the power series:
        \[ e^M = \sum_{k = 0}^{\infty} \frac{M^k}{k!} \]
        Given that $A$ and $B$ commute, we can manipulate the exponential of their sum.
        \[ e^{A + B} = \sum_{k = 0}^{\infty} \frac{(A + B)^k}{k!} \]
        Since $A$ and $B$ commute, we can apply the binomial thm to expand $(A + B)^k$:
        \[ (A + B)^k = \sum_{j = 0}^{k} \begin{pmatrix} k \\ j \end{pmatrix} A^j B^{k - j} \]
        Sub it back, we can get:
        \[ e^{A + B} = \sum_{k = 0}^{\infty} \frac{1}{k!} \sum_{j = 0}^{k} \begin{pmatrix} k \\ j \end{pmatrix} A^j B^{k - j} = \sum_{j = 0}^{\infty} \sum_{k = j}^{\infty} \frac{\begin{pmatrix} k \\ j \end{pmatrix}}{k!} A^j B^{k - j} \]
        Noting that,
        \[ \frac{\begin{pmatrix} k \\ j \end{pmatrix}}{k!} = \frac{1}{j!(k - j)!} \]
        So by rearranging,
        \[ e^{A} e^{B} = \left( \sum_{m = 0}^{\infty} \frac{A^m}{m!} \right) \left( \sum_{mn= 0}^{\infty} \frac{B^n}{n!} \right) = \sum_{j = 0}^{\infty} \sum_{k = j}^{\infty} \frac{\begin{pmatrix} k \\ j \end{pmatrix}}{k!} A^j B^{k - j} = e^{A + B} \]
    \end{proof}
    
    \subsubsection*{(b)}
    
    Let \( A \) and \( B \) be the \( 2 \times 2 \) matrices:
    \[
        A = \begin{bmatrix}
        0 & 1 \\ 0 & 0
        \end{bmatrix}, \quad
        B = \begin{bmatrix}
        0 & 0 \\ 1 & 0
        \end{bmatrix}.
    \]
    Verify \( AB \neq BA \)
    \[
        AB = \begin{bmatrix} 1 & 0 \\ 0 & 0 \end{bmatrix}, \quad
        BA = \begin{bmatrix} 0 & 0 \\ 0 & 1 \end{bmatrix}.
    \]
    Since \( AB \neq BA \), \( A \) and \( B \) do not commute.
    \\
    Noting that both \( A \) and \( B \) are nilpotent (\( A^2 = 0 \), \( B^2 = 0 \)),
    \[
        e^{A} = I + A = \begin{bmatrix}
        1 & 1 \\ 0 & 1
        \end{bmatrix}, \quad
        e^{B} = I + B = \begin{bmatrix}
        1 & 0 \\ 1 & 1
        \end{bmatrix}.
    \]
    Now, we compute \( e^{A} e^{B} \)
    \[
        e^{A} e^{B} = \begin{bmatrix}
        1 & 1 \\ 0 & 1
        \end{bmatrix}
        \begin{bmatrix}
        1 & 0 \\ 1 & 1
        \end{bmatrix}
        = \begin{bmatrix}
        2 & 1 \\ 1 & 1
        \end{bmatrix}.
    \]
    Then, we compute \( e^{A+B} \)
    \[
        A + B = \begin{bmatrix}
        0 & 1 \\ 1 & 0
        \end{bmatrix}, \quad
        (A+B)^2 = I.
    \]
    Using the series expansion of the exponential, by definition, we can get:
    \[
        e^{A+B} = \cosh(1) \cdot I + \sinh(1) \cdot (A+B).
    \]
    Numerical approximation (\( \cosh(1) \approx 1.5431 \), \( \sinh(1) \approx 1.1752 \)):
    \[
        e^{A+B} \approx \begin{bmatrix}
        1.5431 & 1.1752 \\ 1.1752 & 1.5431
        \end{bmatrix}.
    \]
    But, by comparison
    \[
        e^{A} e^{B} = \begin{bmatrix}
        2 & 1 \\ 1 & 1
        \end{bmatrix} \neq
        e^{A+B} \approx \begin{bmatrix}
        1.5431 & 1.1752 \\ 1.1752 & 1.5431
        \end{bmatrix}.
    \]
    Which match our expectation.

\end{document}