\documentclass{article}
\usepackage{graphicx} % Required for inserting images
\usepackage[utf8]{inputenc}
\usepackage{amsmath}
\usepackage{graphicx}
\usepackage{tikz}
\usepackage{array}
\usepackage{amssymb}
\newcommand*{\twoheadrightarrowtail}{\mathrel{\rightarrowtail\kern-1.9ex\twoheadrightarrow}}
% Alternative which doesn't look as good using the normal size, but might work better with smaller sizes too:
%\newcommand*{\twoheadrightarrowtail}{\mathrel{\rlap{$\rightarrowtail$}\twoheadrightarrow}}
\usepackage{amssymb}
\usepackage{amsthm}
\usepackage{multirow}
\usepackage{dcolumn}
\newcolumntype{2}{D{.}{}{2.0}}

\title{MATH 280 HW1}
\author{Hanzhang Yin}
\date{Aug/29/2024}

\begin{document}

\maketitle

\section*{Question 1}

\begin{proof}
    Define the function $h(x) = \frac{f(x)}{g(x)}$, by the def. of the derivative with limit process:
    \[ h'(a) = \lim_{x \rightarrow a} = \frac{h(x) - h(a)}{x - a} \]
    \\
    Expanding $h(x)$ and $h(a)$:
    \[ h'(a) =  \lim_{x \rightarrow a} = \frac{\frac{f(x)}{g(x)} - \frac{f(a)}{g(a)}}{x - a} \]
    \[ \Rightarrow h'(a) = \lim_{x \rightarrow a} \frac{f(x)g(a) - f(a)g(x)}{g(x)g(a)(x - a)} \]
    \\
    Applying $f(x) \approx f(a) + f'(a)(x - a) + E_f(x)$ and $g(x) \approx g(a) + g'(a)(x - a) + E_g(x)$, we can rearrange numerator as:
    \[ f(x)g(a) - f(a)g(x) \approx (f(a) + f'(a)(x - a) + E_f(x))g(a) + f(a)(g(a) + g'(a)(x - a) + E_g(x)) \]
    \[ = f(a)g(a) + f'(a)g(a)(x - a) + E_f(x)g(a) - f(a)g(a) - f(a)g'(a)(x - a) - f(a)E_g(x) \]
    \[ = f'(a)g(a)(x - a) - f(a)g'(a)(x - a) + E_f(x)g(a) - f(a)E_g(x) \]
    \\
    Applying the definition of $E_f(x)$ and $E_g(x)$:
    \[ \lim_{x \rightarrow a} \frac{E_f(x)g(a) - f(a)E_g(x)}{g(x)g(a)g(x - a)} = 0 \]
    \\
    we can get overall:
    \[ h'(a) = \lim_{x \rightarrow a} \frac{f'(a)g(a)(x - a) - f(a)g'(a)(x - a) + E_f(x)g(a) - f(a)E_g(x)}{g(x)g(a)g(x - a)} \]
    \\
    Since $g(x) \approx g(a)$ when $x = a$:
    \[ \Rightarrow \left( \frac{f}{g} \right)^{'}  = h'(a) = \lim_{x \rightarrow a} \frac{f'(a)g(a) - f(a)g'(a)}{g(a)^2} \]
\end{proof}

\section*{Question 2}

\subsection*{(a)}
Given,
\begin{itemize}
    \item $f(x) = 2 - (x - 3) + E_f(x)$, near $x = 3$, with $|E_f(x) < 1|$ one the interval $[2,4]$.
    \item $g(x) = x^3 - 2x^2 + 3$; $g(2) = 8 - 4(2) + 3 = 3, g'(2) = 12 - 8 = 4$.
\end{itemize}

\begin{proof}
    From the initial assumptions, here is the approximation of $f(x)$ near $x = 3$: 
    \[f(x) = 2 - (x - 3) + E_f(x)\]
    \\
    Value of $g$ at $f(3)$:
    \[ f(3) = 2 - (x3- 3) + E_f(3) = 2 + E_f(3) \]
    \\
    Noticing that $|E_f(x)| < 1$, then
    \[ g(f(3)) = g(2 + E_f(3)) = g(2) = 3 \]
    \\
    Then, we can use the Taylor Expansion around $f(3) = 2$. 
    The first order Taylor expansion for $g(f(x))$ around $x = 3$ would be:
    \[ g(f(x)) \approx g(2) + g'(2) \cdot (f(x) - 2) \]
    Then we can sub $f(x)$ in:
    \[ g(f(x)) \approx 3 + 4(2 + E_f(x) - (x - 3) - 2) \]
    \[ = 3 + 4(-x + 3 + E_f(x)) \]
    \[ = 3 + 4(-x + 3) + 4E_f(x) \]
    \[ = 15 - 4x + 4E_f(x) \]
    \\
    Given $E(x) << (x - a)$  and considering the linearity of \( g'(x): E_{f \circ g} = 4E_f(x) \).
    As $E_f(x)$ is much smaller $(x - 3)$, and given $|E_f(x)| < 1: |E_{f \circ g}| \leq 4$.
\end{proof}

\newpage

\subsection*{(b)}
Given $h(x)$ is differentiable at $x = 0$ and $h(0) = 3$, WTS       the bound the
error of $f(h(x))$ near zero.
\\
\begin{proof}
    To precisely bound the error of $f(h(x))$ near zero, we need two more information:
    \begin{itemize}
        \item Explicit form of $E_f(x)$
        \item Rate of Change of $h(x)$
    \end{itemize}

    By knowing these two conditions, we can bound $E_f(h(x))$ by the following approach:
    \\
    By definition of Taylor Expansion, we can subsitute
    \[ h(x) = 3 + h'(0)x + o(x) \]
    \\
    Where $o(x)$ is the higher order terms that become negligibly small faster than linear.
    \\
    Then, suppose $E_f(x) = k(x - 3)^n$, then:
    \[ E_f(h(x)) = k(h(x) - 3)^n = k(h'(0)x + o(x))^n \]
    \\
    For evaluation we will only focus on how the ``linear'' terms in $x$ contribute to $E_f(h(x))$.
    \\
    Assume $E_f(x) \sim (x - 3)^n$ and $h(x) = 3 + h'(0)x + o(x)$, the bound of $E_f(h(x))$ when $x = 0$ can be written in the form:
    \[ |E_f(h(x))| \leq |k| \cdot |h'(0)x + o(x)|^n \]

\end{proof}

\section*{Question 3}
Given, $P(t)$ s.t. $P(1) = 4, P'(t) = tP(t)^2$

\subsection*{(a) Quadratic Approximation}
\begin{proof}
    Given,
    \[ P'(1) = 4, P'(t) = tP(t)^2 \]
    For $P^{'}(t)$ and $P^{''}(t)$, we can get (using the product rule):
    \\
    \[ P'(1) = 1 \times 4^2 = 16 \]
    \[ P^{''}(t) = \frac{d}{dt}[tP(t)^2] = P(t)^2 + 2tP(t)P^{'}(t)\]
    \\
    Sub $t = 1$ into $P^{''}(t)$:
    \[ P^{''}(t) = 4^2 + 2 \times 1 \times 4 \times 16 = 16 + 128 = 144 \]
    \\
    Then we can apply the Taylor Expansion around $t = 1$, include the small error term $E(t-1)$ s.t. $E(t-1) << t-1$:
    \[ P(t) \approx P(1) + P'(1)(t - 1) + \frac{1}{2}P^{''}(1)(t - 1)^2 + E(t - 1) \]
    \[ P(t) \approx 4 + 16(t - 1) + 72(t - 1)^2 + E(t - 1) \]
    \\
    Then, we can estimate $P(0)$ and $P(-1)$. Subsitute $t = 0$ and $t = -1$, we can get:
    \[ P(0) = 4 + 16(-1) + 72(1)^2 = 4 - 16 + 72 = 60 \]
    \[ P(-1) = 4 + 16(-2) + 72(4) = 4 - 32 + 288 = 260 \]
    \\
    Noting that $E(t - 1)$ is negligible near $t = 1$.
\end{proof}

\subsection*{(b) Analysis Critical Point at $t = 0$}
Noting that in here we can apply the 2nd derivative test.
\begin{proof}
    From (a), we know that,
    \[ P^{''}(0) = P(0)^2 \approx 60^2 = 3600 \]
    \\
    Since $P^{''}(t) > 0$ for $t = 0$, critical point at $t = 0$ is a local minimum point.
\end{proof}

\subsection*{(c)}
I think in this case the estimation using quadratic approximation for $P(0)$ and $P(-1)$ might likely underestimate the exact values. Such phenomenon aries is 
because of the quadratic model might not fully capture the exponential growth.
Specifically, since $P'(t) = tP(t)^2$, the rate of change of $P(t)$ depends on the product of $t$ and $P(t)^2$. This might lead to rapid changes in the function's value, 
particularly when $t \ \& \ P(t)$ are significantly large.
Hence, quadratic approximation might fails to accurately predict the behavior of the function $P(t)$ and underestimate its actual value due to its lack of complexity.

\end{document}