\documentclass{article}
\usepackage[margin=1in]{geometry}
\usepackage{amsmath, amsthm, amssymb, amsfonts}
\usepackage{graphicx}
\usepackage{xcolor}
\usepackage{enumerate}
\usepackage{environ}

\newcommand{\inv}{^{-1}}

\newcounter{pnum}
\NewEnviron{problem}{
    \stepcounter{pnum}
    \begin{center}
        \fbox{
        \begin{minipage}{0.95\linewidth}
            \textbf{\thepnum.} \BODY
        \end{minipage}}
    \end{center}

    \textbf{Solution. }
}

\title{\vspace{-4em}Homework 6 (root finding)}
\author{Hanzhang Yin}
\begin{document}

\maketitle


\begin{problem}
    One view of the secant method: it is a coarser Newton's method. We've seen that it has some of the speed of Newton's method. One might also hope that it enjoys similar convergence properties.\\[-0.5em]
    
    Adapt the convergence proof for Newton's method to show that the secant method also always converges under the following assumptions about the function \(f\) on the interval \([a,b]\):
    \begin{enumerate}[\hspace{2em} i)]
        \item \(f\) is twice continuously differentiable
        \item \(f' > 0\)
        \item \(f''> 0\)
        \item \(f\) has a root \(x\) in the interval
        \item the two initial guesses \(x_0,x_1\) are both to the right of the root.
    \end{enumerate}

    Hint: you will have to use convexity in a slightly more interesting way than in NM -- the graph of \(f\) does not lie above the secant line, but you can argue that the right (well, left!) piece still does.
\end{problem}

\begin{problem}
    Another view of the secant method, discussed in class, is as a weighted bisection method. Here too, one might hope for a convergence guarantee, because BM is much more robust than NM in that regard.\\[-0.5em]

    Consider a modified secant method which at step \(k\) takes in endpoints \(a_k,b_k\), calculates their weighted midpoint \(c_k\) and then returns two new endpoints \(a_{k+1},b_{k+1}\), one of which is \(c_k\), to which IVT applies. These new endpoints are input to the next step.\\[-0.5em]

    Prove that if \(f\) is continuous on \([a,b] = [a_0,b_0]\) and the IVT applies to \(f\) on the interval, then the sequence \(c_k\) from the modified secant method converges to a root of \(f\).\\[-0.5em]

    Hint: the reason for convergence is \emph{not} the same as for bisection. This would require the stronger assumption that \(f\) is continuously differentiable. In fact:

    [Bonus] Give an example where the sequences \(x_k\) and \(y_k\) converge to different points, so squeeze does not apply.
\end{problem}


\begin{problem}
    Suppose \(f(x)\) and \(g(x)\) are functions with a common root \(x=a\).
    \begin{enumerate}[\hspace{2em}a)]
        \item Prove that a solution to the homotopy continuation initial value problem
    \[x'(t) = -\frac{H_t}{H_x}\ \ \ \ x(0) = a\]
    is the constant function \(x=a\).
        \item Give an example where the solution above is \emph{not} unique.
    \end{enumerate}

    Hint: see handout for a picture of (a). Think about how it could be adapted (b); you can even use the tool to help you construct an example.
\end{problem}

\end{document}
