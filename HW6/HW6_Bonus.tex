\documentclass{article}
\usepackage[margin=1in]{geometry}
\usepackage{amsmath, amsthm, amssymb, amsfonts}
\usepackage{graphicx}
\usepackage{xcolor}
\usepackage{enumerate}
\usepackage{environ}

\newcommand{\inv}{^{-1}}

\newcounter{pnum}
\NewEnviron{problem}{
    \stepcounter{pnum}
    \begin{center}
        \fbox{
        \begin{minipage}{0.95\linewidth}
            \textbf{\thepnum.} \BODY
        \end{minipage}}
    \end{center}

    \textbf{Solution. }
}

\title{\vspace{-4em}Homework 6 Bonus}
\author{Hanzhang Yin}
\begin{document}

\maketitle

\subsection*{Q2 Bonus}

\textbf{Bonus: Example Where \( \{ a_k \} \) and \( \{ b_k \} \) Converge to Different Points}
\\
We provide an example where the sequences \( \{ a_k \} \) and \( \{ b_k \} \) converge to different points, so the squeeze theorem does not apply.
\\
\textbf{Example:}
\\
Let \( f: [0, 3] \to \mathbb{R} \) be defined by
\[
f(x) = \begin{cases}
(x - 1), & \text{for } x \in [0, 2], \\
(x - 2), & \text{for } x \in (2, 3].
\end{cases}
\]
Note that \( f \) is continuous on \( [0, 3] \), except at \( x = 2 \), but let's adjust the function to be continuous at \( x = 2 \). Define
\[
f(x) = \begin{cases}
(x - 1), & \text{for } x \in [0, 2], \\
0, & \text{for } x = 2, \\
(x - 2), & \text{for } x \in (2, 3].
\end{cases}
\]
Now, \( f \) is continuous on \( [0, 3] \).
\\
Let the initial interval be \( [a_0, b_0] = [0.5, 2.5] \).
Compute \( f(a_0) = -0.5 \), \( f(b_0) = 0.5 \), so \( f(a_0) \cdot f(b_0) < 0 \).
At each iteration, the modified secant method computes
\[
c_k = b_k - f(b_k) \cdot \frac{b_k - a_k}{f(b_k) - f(a_k)}.
\]
Because of the piecewise linear nature of \( f \), the secant method will alternate between intervals that include \( x = 1 \) and \( x = 2 \), the roots of \( f \).
\\
\textbf{Iteration Steps:}

\begin{enumerate}
  \item First Iteration:
  \\
  \( a_0 = 0.5 \), \( b_0 = 2.5 \), \( f(a_0) = -0.5 \), \( f(b_0) = 0.5 \).
  Compute \( c_0 \):
  \[
    c_0 = 2.5 - 0.5 \cdot \frac{2.5 - 0.5}{0.5 - (-0.5)} = 2.5 - 0.5 \cdot \frac{2}{1} = 2.5 - 1 = 1.5.
  \]
  \( f(c_0) = 1.5 - 1 = 0.5 > 0 \). Since \( f(a_0) \cdot f(c_0) < 0 \), set \( a_1 = a_0 = 0.5 \), \( b_1 = c_0 = 1.5 \).
  \item Second Iteration:
  \\
  \( a_1 = 0.5 \), \( b_1 = 1.5 \), \( f(a_1) = -0.5 \), \( f(b_1) = 0.5 \).
  Compute \( c_1 \):
  \[
    c_1 = 1.5 - 0.5 \cdot \frac{1.5 - 0.5}{0.5 - (-0.5)} = 1.5 - 0.5 \cdot \frac{1}{1} = 1.5 - 0.5 = 1.0.
  \]
  \( f(c_1) = 1.0 - 1 = 0 \). Since \( f(c_1) = 0 \), we have found a root at \( x = 1 \).
\end{enumerate}

However, suppose that due to rounding errors or different initial intervals, the method alternates between intervals converging to \( x = 1 \) and \( x = 2 \).
\\
In this example, the sequences \( \{ a_k \} \) and \( \{ b_k \} \) can converge to different roots of \( f \), so the squeeze theorem does not apply directly. Despite this, the sequence \( \{ c_k \} \) generated by the modified secant method still converges to a root of \( f \).

\end{document}