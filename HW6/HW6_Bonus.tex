\documentclass{article}
\usepackage[margin=1in]{geometry}
\usepackage{amsmath, amsthm, amssymb, amsfonts}
\usepackage{graphicx}
\usepackage{xcolor}
\usepackage{enumerate}
\usepackage{environ}

\newcommand{\inv}{^{-1}}

\newcounter{pnum}
\NewEnviron{problem}{
    \stepcounter{pnum}
    \begin{center}
        \fbox{
        \begin{minipage}{0.95\linewidth}
            \textbf{\thepnum.} \BODY
        \end{minipage}}
    \end{center}

    \textbf{Solution. }
}

\title{\vspace{-4em}Homework 6 Bonus}
\author{Hanzhang Yin}
\begin{document}

\maketitle

\subsubsection*{Q2 Bonus}

\textbf{Bonus: Example Where \( \{ a_k \} \) and \( \{ b_k \} \) Converge to Different Points}
\\
We provide an example of a continuous function \( f \) and initial intervals \( [a_0, b_0] \) and \( [c_0, d_0] \) where the sequences \( \{ a_k \} \) and \( \{ b_k \} \) converge to different roots.
\\
\textbf{Bonus Example:}
\\
Let \( f: [0, 4] \to \mathbb{R} \) be defined by
\[
    f(x) = (x - 1)(x - 2)(x - 3).
\]
Note that \( f \) is continuous on \([0, 4]\) and has roots at \( x = 1 \), \( x = 2 \), and \( x = 3 \).
\\
\textbf{Sequence 1: Converging to \( x = 1 \)}
\\
Choose initial interval \( [a_0, b_0] = [0.5, 1.5] \).
\[
    f(a_0) = (0.5 - 1)(0.5 - 2)(0.5 - 3) = (-0.5)(-1.5)(-2.5) = -1.875 < 0,
\]
\[
    f(b_0) = (1.5 - 1)(1.5 - 2)(1.5 - 3) = (0.5)(-0.5)(-1.5) = 0.375 > 0.
\]
Since \( f(a_0) \cdot f(b_0) < 0 \), there is a root in \( [0.5, 1.5] \), specifically at \( x = 1 \).
\\
Applying the secant method:
\[
    c_0 = 1.5 - 0.375 \cdot \frac{1.5 - 0.5}{0.375 - (-1.875)} = 1.5 - 0.375 \cdot \frac{1}{2.25} = 1.5 - 0.375 \cdot 0.4444 \approx 1.5 - 0.1667 = 1.3333,
\]
\[
    f(c_0) \approx f(1.3333) = (1.3333 - 1)(1.3333 - 2)(1.3333 - 3) \approx (0.3333)(-0.6667)(-1.6667) \approx 0.3704 > 0.
\]
Since \( f(c_0) > 0 \), update the interval to \( [a_1, b_1] = [0.5, 1.3333] \).
\\
Continuing this process, the sequence \( \{ a_k \} \) will converge to \( x = 1 \).
\\
\textbf{Sequence 2: Converging to \( x = 3 \)}
\\
Choose initial interval \( [c_0, d_0] = [2.5, 3.5] \).
\[
    f(c_0) = (2.5 - 1)(2.5 - 2)(2.5 - 3) = (1.5)(0.5)(-0.5) = -0.375 < 0,
\]
\[
    f(d_0) = (3.5 - 1)(3.5 - 2)(3.5 - 3) = (2.5)(1.5)(0.5) = 1.875 > 0.
\]
Since \( f(c_0) \cdot f(d_0) < 0 \), there is a root in \( [2.5, 3.5] \), specifically at \( x = 3 \).
\\
Applying the secant method:
\[
    e_0 = 3.5 - 1.875 \cdot \frac{3.5 - 2.5}{1.875 - (-0.375)} = 3.5 - 1.875 \cdot \frac{1}{2.25} = 3.5 - 1.875 \cdot 0.4444 \approx 3.5 - 0.8333 = 2.6667,
\]
\[
    f(e_0) \approx f(2.6667) = (2.6667 - 1)(2.6667 - 2)(2.6667 - 3) \approx (1.6667)(0.6667)(-0.3333) \approx -0.3704 < 0.
\]
Since \( f(e_0) < 0 \), update the interval to \( [c_1, d_1] = [2.6667, 3.5] \).
\\
Continuing this process, the sequence \( \{ b_k \} \) will converge to \( x = 3 \).
\\
\textbf{Conclusion:}
\\
The sequence \( \{ a_k \} \) generated from the initial interval \( [0.5, 1.5] \) converges to the root at \( x = 1 \).
The sequence \( \{ b_k \} \) generated from the initial interval \( [2.5, 3.5] \) converges to the root at \( x = 3 \).
\\
by choosing different initial intervals bracketing distinct roots, the sequences \( \{ a_k \} \) and \( \{ b_k \} \) can converge to different points.
\end{document}