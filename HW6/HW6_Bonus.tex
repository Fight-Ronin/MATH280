\documentclass{article}
\usepackage[margin=1in]{geometry}
\usepackage{amsmath, amsthm, amssymb, amsfonts}
\usepackage{graphicx}
\usepackage{xcolor}
\usepackage{enumerate}
\usepackage{environ}

\newcommand{\inv}{^{-1}}

\newcounter{pnum}
\NewEnviron{problem}{
    \stepcounter{pnum}
    \begin{center}
        \fbox{
        \begin{minipage}{0.95\linewidth}
            \textbf{\thepnum.} \BODY
        \end{minipage}}
    \end{center}

    \textbf{Solution. }
}

\title{\vspace{-4em}Homework 6 Bonus}
\author{Hanzhang Yin}
\begin{document}

\maketitle

\subsubsection*{Q2 Bonus}

\textbf{Bonus: Example Where \( \{ a_k \} \) and \( \{ b_k \} \) Converge to Different Points}
    \\
    We provide an example of a continuous function \( f \) and initial interval \( [a_0, b_0] \) where the sequences \( \{ a_k \} \) and \( \{ b_k \} \) converge to different points.
    \\
    \textbf{Bonus Example:}
    \\
    Let \( f: [0, 2] \to \mathbb{R} \) be defined by
    \[
        f(x) = (x - 1)(x - 1.5).
    \]
    Note that \( f \) is continuous on \( [0, 2] \) and has roots at \( x = 1 \) and \( x = 1.5 \).
    \\
    Choose initial endpoints \( a_0 = 0 \) and \( b_0 = 2 \). Then \( f(a_0) = (0 - 1)(0 - 1.5) = ( -1)( -1.5) = 1.5 > 0 \), and \( f(b_0) = (2 - 1)(2 - 1.5) = (1)(0.5) = 0.5 > 0 \). Since \( f(a_0) \cdot f(b_0) > 0 \), we need to adjust the initial interval to ensure that \( f(a_0) \cdot f(b_0) < 0 \).
    \\
    Let's instead choose \( a_0 = 0.5 \) and \( b_0 = 2 \). Then \( f(a_0) = (0.5 - 1)(0.5 - 1.5) = ( -0.5)( -1) = 0.5 > 0 \), and \( f(b_0) = 0.5 > 0 \), so again the signs are the same.
    \\
    Adjust again by choosing \( a_0 = 1 \) and \( b_0 = 1.75 \). Then
    \[
        f(a_0) = (1 - 1)(1 - 1.5) = 0 \cdot ( -0.5) = 0,
    \]
    \[
        f(b_0) = (1.75 - 1)(1.75 - 1.5) = (0.75)(0.25) = 0.1875 > 0.
    \]
    Still, \( f(a_0) \cdot f(b_0) = 0 \), so we need to find an interval where \( f \) changes sign.
    \\
    Consider \( a_0 = 0.75 \) and \( b_0 = 1.25 \). Then
    \[
        f(a_0) = (0.75 - 1)(0.75 - 1.5) = ( -0.25)( -0.75) = 0.1875 > 0,
    \]
    \[
        f(b_0) = (1.25 - 1)(1.25 - 1.5) = (0.25)( -0.25) = -0.0625 < 0.
    \]
    Thus, \( f(a_0) \cdot f(b_0) < 0 \).
    \\
    Now, apply the modified secant method:
    \\
    At each step, compute
    \[
      c_k = b_k - f(b_k) \cdot \frac{b_k - a_k}{f(b_k) - f(a_k)}.
    \]
    Select \( a_{k+1} \) and \( b_{k+1} \) based on the sign of \( f(c_k) \).
    \\
    Since \( f \) has roots at \( x = 1 \) and \( x = 1.5 \), and \( f(a_0) > 0 \) while \( f(b_0) < 0 \), the method will converge to the root at \( x = 1 \).
    \\
    However, due to the shape of \( f \), the sequences \( \{ a_k \} \) and \( \{ b_k \} \) may converge to different points. Specifically, \( \{ a_k \} \) may converge to \( x = 1 \), while \( \{ b_k \} \) may converge to \( x = 1.25 \), depending on the behavior of \( f \) and the secant method updates.
    \\
    This example illustrates that \( \{ a_k \} \) and \( \{ b_k \} \) need not converge to the same point, so the squeeze theorem cannot be directly applied to \( \{ c_k \} \) via \( \{ a_k \} \) and \( \{ b_k \} \).

\end{document}