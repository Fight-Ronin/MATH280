\documentclass{article}
\usepackage[margin=1in]{geometry}
\usepackage{amsmath, amsthm, amssymb, amsfonts}
\usepackage{graphicx}
\usepackage{xcolor}
\usepackage{enumerate}

\newcommand{\inv}{^{-1}}

\newcounter{pnum}
\newcommand\problem[1]{\stepcounter{pnum}\begin{center}\fbox{\begin{minipage}{0.95\linewidth}
\textbf{\thepnum.} #1
\end{minipage}}\end{center}}

\newcommand\solution{\vspace{0.4em}\textbf{Solution.}}

\title{Homework 4 (interpolation)}
\author{name}
\begin{document}

\maketitle


\problem{Let \(M\) be a Markov matrix (sum of entries in a column is 1) with all diagonal entries nonzero. Show that the only possible eigenvalue with norm \(1\) is \(1\), and that any other eigenvalue has strictly smaller norm. Hint: apply the gershgorin circle theorem to \(M^T\).}
\solution

\begin{proof}

Let \( M \) be an \( n \times n \) Markov matrix with all diagonal entries satisfying \( M_{ii} > 0 \). We aim to show that:
\begin{enumerate}
    \item The only eigenvalue of \( M \) with absolute value \( 1 \) is \( 1 \).
    \item All other eigenvalues \( \lambda \) satisfy \( |\lambda| < 1 \).
\end{enumerate}

Since \( M \) is column stochastic, its transpose \( M^T \) is row stochastic:
\[
\sum_{j=1}^n M^T_{ij} = \sum_{j=1}^n M_{ji} = 1 \quad \forall i = 1, 2, \dots, n.
\]
Moreover, the diagonal entries of \( M^T \) satisfy \( M^T_{ii} = M_{ii} > 0 \).
\\
The Gershgorin Circle Theorem states that every eigenvalue \( \lambda \) of \( M^T \) lies within at least one Gershgorin disc \( D_i \) defined for each row \( i \):
\[
D_i = \left\{ \lambda \in \mathbb{C} \ \bigg| \ |\lambda - M^T_{ii}| \leq R_i \right\},
\]
where \( R_i \) is the sum of the absolute values of the non-diagonal entries in row \( i \):
\[
R_i = \sum_{\substack{j \neq i}}^n |M^T_{ij}| = \sum_{\substack{j \neq i}}^n M_{ji} = 1 - M_{ii}.
\]
Thus, each Gershgorin disc \( D_i \) is centered at \( M_{ii} \) with radius \( 1 - M_{ii} \):
\[
D_i = \left\{ \lambda \in \mathbb{C} \ \bigg| \ |\lambda - M_{ii}| \leq 1 - M_{ii} \right\}.
\]
Suppose \( \lambda \) is an eigenvalue of \( M^T \) (and hence of \( M \)) with \( |\lambda| = 1 \). Since \( \lambda \) lies within some Gershgorin disc \( D_i \):
\[
|\lambda - M_{ii}| \leq 1 - M_{ii}.
\]
Squaring both sides:
\[
|\lambda - M_{ii}|^2 \leq (1 - M_{ii})^2.
\]
Expanding the left side using \( |\lambda|^2 = 1 \):
\[
|\lambda - M_{ii}|^2 = |\lambda|^2 - 2 \text{Re}(\lambda) M_{ii} + M_{ii}^2 = 1 - 2 \text{Re}(\lambda) M_{ii} + M_{ii}^2.
\]
Setting this less than or equal to the right side:
\[
1 - 2 \text{Re}(\lambda) M_{ii} + M_{ii}^2 \leq 1 - 2 M_{ii} + M_{ii}^2.
\]
Subtracting \( 1 + M_{ii}^2 \) from both sides:
\[
-2 \text{Re}(\lambda) M_{ii} \leq -2 M_{ii}.
\]
Dividing by \( -2 M_{ii} \) (note that \( M_{ii} > 0 \) reverses the inequality):
\[
\text{Re}(\lambda) \geq 1.
\]
However, since \( |\lambda| = 1 \), the maximum possible value of \( \text{Re}(\lambda) \) is \( 1 \), achieved only if \( \lambda = 1 \).

\begin{enumerate}
    \item \textbf{Uniqueness of Eigenvalue 1}: The only eigenvalue \( \lambda \) with \( |\lambda| = 1 \) must satisfy \( \lambda = 1 \).
    \item \textbf{All Other Eigenvalues}: Any other eigenvalue \( \lambda \neq 1 \) must lie strictly inside the unit circle, i.e., \( |\lambda| < 1 \).
\end{enumerate}
\\
Therefore, \( 1 \) is the sole eigenvalue of \( M \) with absolute value \( 1 \), and all other eigenvalues have strictly smaller magnitudes.
\end{proof}

\problem{[Book 6.4.14] Determine whether the following is a natural cubic spline:
\[f(x) = \begin{cases}
    2(x+1)^3 + (x+1)^3 & x \in [-1,0]\\
    3 + 5x + 3x^2 & x \in [0,1]\\
    11 + 11(x-1) + 3(x-1)^2 - (x-1)^3 & x \in [1,2]
\end{cases}\]}
\solution

\begin{proof}
    Simplification of Each Piece:
    \begin{enumerate}
        \item For $x \in [-1, 0]$:
        \[ f(x) = 2(x + 1)^3 + (x + 1)^3 = 3(x + 1)^3 \]
        \item For $x \in [0, 1]$:
        \[ f(x) = 3 + 5x + 3x^2 \]
        \item For $x \in [1, 2]$:
        \[ f(x) = 11 + 11(x - 1) + 3(x = 1)^2 - (x - 1)^3 \]
        \[ = 11 + 11x -11 + 3(x^2 - 2x + 1) - (x^3 - 3x^2 + 3x - 1) \]
        \[ = -x^3 + 6x^2 + 2x + 4 \]
    \end{enumerate}
    \\
    Check Continuity at the Knots $x = 0$ and $x = 1$.
    \\
    At $x = 0$:
    \begin{itemize}
        \item From the left ($x \rightarrow 0^{-}$): $f(0^{-}) = 3(0)^3 + 9(0)^2 + 9(0) + 3 = 3$
        \item From the left ($x \rightarrow 0^{+}$): $f(0^{+}) = 3 + 5(0) + 3(0)^2 = 3$
        \item $f$ is continuous at $x = 0$
    \end{itemize}
    At $x = 1$:
    \begin{itemize}
        \item From the left ($x \rightarrow 1^{-}$): $f(1^{-}) = 3 + 5(1) + 3(1)^2 = 11$
        \item From the left ($x \rightarrow 1^{+}$): $f(1^{+}) = -1 + 6(1)^2 + 2(1) + 4 = 11$
        \item $f$ is continuous at $x = 1$
    \end{itemize}
    \\
    Check Differentiability at the Knots:
    \\
    Compute the first derivative $f^{'}(0)$ in each interval:
    \begin{itemize}
        \item $x \in [-1, 0]: f^{'}(x) = 9x^2 + 18x + 9$
        \item $x \in [0, 1]: f^{'}(x) = 5 + 6x$
        \item $x \in [1, 2]: f^{'}(x) = -3x^2 + 12x + 2$
    \end{itemize}
    At $x = 0: $
    \begin{itemize}
        \item From the left: $f^{'}(0^{-}) = 9(0)^2 + 18(0) + 9 = 9$
        \item From the left: $f^{'}(0^{+}) = 5 + 6(0) = 5$
        \item The derivative are not equal; $f^{'}(x)$ is not countinuous at $x = 0$
    \end{itemize}
    \\
    Since the first derivative $f^{'}(x)$ is not continuous at $x = 0$, the function $f(x)$ is not differentiable at 
    that point. This violates the requirement for a spline to be twice continuously differentiable over the interval.
    Therefore, \textbf{the given function is not a natural cubic spline.}
\end{proof}


\problem{[Book 6.4.25] Determine coefficients \(a,b,c,d\), which make the following a cubic spline:
\[f(x) = \begin{cases}
    x^3 & -1\leq x \leq 0\\
    a+bx+cx^2 + dx^3 & 0\leq x \leq 1
\end{cases}\]}
\solution

\problem{Let \(f(x) = \arctan(x)\) 
\begin{enumerate}[a)]
    \item Suppose you interpolated \(f(x)\) by a degree \(3\) polynomial using the Chebyshev nodes as \(x\) values [you do not need to calculate the interpolating polynomial]. Estimate the error associated to this interpolation.
    \item Using a taylor series around \(0\), write down a degree \(5\) approximation to \(f(x)\).
    \item With Taylor's form of the remainder, estimate the error associated to the interpolation in (b). (you may use a computer to calculate the 6th derivative, but you must bound it on your own, explaining your work carefully)
    \item Compare your error estimates (a) and (c). Which seems better, and why do you think this might be the case? Hint: taylor series are a little like interpolating just at a single point, using derivatives at just that point to provide extra constraints.
\end{enumerate}
}
\solution

\problem{Determine a quadratic spline approximation \(S(x)\) to \(f(x) = \arctan(x)\) with nodes \(-1,0,1\).}

\problem{Let \(f(x) = 4x^2 - 4^{x}\).
\begin{enumerate}
    \item Using the intermediate value theorem, show that \(f(x)\) has at least one root in \([-1,0]\) and another in \([0,1.5]\).
    \item Interpolate \(f(x)\) by a degree \(3\) polynomial using nodes \(x=-1/2,0,1/2\).
    \item Use the interpolation to estimate the roots of \(f(x)\) in those intervals.
\end{enumerate}}
\solution


\end{document}
