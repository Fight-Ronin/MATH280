\documentclass{article}
\usepackage[margin=1in]{geometry}
\usepackage{amsmath, amsthm, amssymb, amsfonts}
\usepackage{graphicx}
\usepackage{xcolor}
\usepackage{enumerate}

\newcommand{\inv}{^{-1}}

\newcounter{pnum}
\newcommand\problem[1]{\stepcounter{pnum}\begin{center}\fbox{\begin{minipage}{0.95\linewidth}
\textbf{\thepnum.} #1
\end{minipage}}\end{center}}

\newcommand\solution{\vspace{0.4em}\textbf{Solution.}}

\title{Homework 4 (interpolation)}
\author{name}
\begin{document}

\maketitle


\problem{Let \(M\) be a Markov matrix (sum of entries in a column is 1) with all diagonal entries nonzero. Show that the only possible eigenvalue with norm \(1\) is \(1\), and that any other eigenvalue has strictly smaller norm. Hint: apply the gershgorin circle theorem to \(M^T\).}
\solution

\problem{[Book 6.4.14] Determine whether the following is a natural cubic spline:
\[f(x) = \begin{cases}
    2(x+1)^3 + (x+1)^3 & x \in [-1,0]\\
    3 + 5x + 3x^2 & x \in [0,1]\\
    11 + 11(x-1) + 3(x-1)^2 - (x-1)^3 & x \in [1,2]
\end{cases}\]}
\solution



\problem{[Book 6.4.25] Determine coefficients \(a,b,c,d\), which make the following a cubic spline:
\[f(x) = \begin{cases}
    x^3 & -1\leq x \leq 0\\
    a+bx+cx^2 + dx^3 & 0\leq x \leq 1
\end{cases}\]}
\solution

\problem{Let \(f(x) = \arctan(x)\) 
\begin{enumerate}[a)]
    \item Suppose you interpolated \(f(x)\) by a degree \(3\) polynomial using the Chebyshev nodes as \(x\) values [you do not need to calculate the interpolating polynomial]. Estimate the error associated to this interpolation.
    \item Using a taylor series around \(0\), write down a degree \(5\) approximation to \(f(x)\).
    \item With Taylor's form of the remainder, estimate the error associated to the interpolation in (b). (you may use a computer to calculate the 6th derivative, but you must bound it on your own, explaining your work carefully)
    \item Compare your error estimates (a) and (c). Which seems better, and why do you think this might be the case? Hint: taylor series are a little like interpolating just at a single point, using derivatives at just that point to provide extra constraints.
\end{enumerate}
}
\solution

\problem{Determine a quadratic spline approximation \(S(x)\) to \(f(x) = \arctan(x)\) with nodes \(-1,0,1\).}

\problem{Let \(f(x) = 4x^2 - 4^{x}\).
\begin{enumerate}
    \item Using the intermediate value theorem, show that \(f(x)\) has at least one root in \([-1,0]\) and another in \([0,1.5]\).
    \item Interpolate \(f(x)\) by a degree \(3\) polynomial using nodes \(x=-1/2,0,1/2\).
    \item Use the interpolation to estimate the roots of \(f(x)\) in those intervals.
\end{enumerate}}
\solution


\end{document}
