\documentclass{article}
\usepackage[margin=1in]{geometry}
\usepackage{amsmath, amsthm, amssymb, amsfonts}
\usepackage{graphicx}
\usepackage{xcolor}
\usepackage{enumerate}

\newcommand{\inv}{^{-1}}

\newcounter{pnum}
\newcommand\problem[1]{\stepcounter{pnum}\begin{center}\fbox{\begin{minipage}{0.95\linewidth}
\textbf{\thepnum.} #1
\end{minipage}}\end{center}}

\newcommand\solution{\vspace{0.4em}\textbf{Solution.}}

\title{Homework 4 (interpolation)}
\author{Hanzhang Yin}
\begin{document}

\maketitle


\problem{Let \(M\) be a Markov matrix (sum of entries in a column is 1) with all diagonal entries nonzero. Show that the only possible eigenvalue with norm \(1\) is \(1\), and that any other eigenvalue has strictly smaller norm. Hint: apply the gershgorin circle theorem to \(M^T\).}
\solution
\begin{proof}
    *NOTE: $\mathbb{C}$: Complex Number, and $Re()$: Real Number part.
    \\
    Let \( M \) be an \( n \times n \) Markov matrix with all diagonal entries \( M_{ii} > 0 \). Since \( M \) is column stochastic, its transpose \( M^T \) is row stochastic, i.e., \( \sum_{j=1}^n M^T_{ij} = 1 \) for all \( i \).
    By the Gershgorin Circle Theorem, every eigenvalue \( \lambda \) of \( M^T \) (and hence of \( M \)) lies within at least one disc \( D_i \) centered at \( M_{ii} \) with radius \( R_i = 1 - M_{ii} \):
    \[
        D_i = \{ \lambda \in \mathbb{C} : |\lambda - M_{ii}| \leq 1 - M_{ii} \}.
    \]
    Suppose \( \lambda \) is an eigenvalue with \( |\lambda| = 1 \). Then,
    \[
        |\lambda - M_{ii}| \leq 1 - M_{ii}.
    \]
    But since \( |\lambda| = 1 \) and \( M_{ii} > 0 \),
    \[
        |\lambda - M_{ii}| \geq |\lambda| - M_{ii} = 1 - M_{ii}.
    \]
    Thus,
    \[
        |\lambda - M_{ii}| = 1 - M_{ii},
    \]
    which means \( \lambda \) lies on the boundary of \( D_i \). Expanding,
    \begin{align*}
        |\lambda - M_{ii}|^2 &= (1 - M_{ii})^2, \\
        |\lambda|^2 - 2 M_{ii} \operatorname{Re}(\lambda) + M_{ii}^2 &= 1 - 2 M_{ii} + M_{ii}^2, \\
    1 - 2 M_{ii} \operatorname{Re}(\lambda) + M_{ii}^2 &= 1 - 2 M_{ii} + M_{ii}^2.
    \end{align*}
    Subtracting \( 1 - 2 M_{ii} + M_{ii}^2 \) from both sides yields:
    \[
        -2 M_{ii} \operatorname{Re}(\lambda) = -2 M_{ii}.
    \]
    Since \( M_{ii} > 0 \), dividing both sides by \( -2 M_{ii} \) gives:
    \[
        \operatorname{Re}(\lambda) = 1.
    \]
    With \( |\lambda| = 1 \) and \( \operatorname{Re}(\lambda) = 1 \), it follows that \( \lambda = 1 \).
    Therefore, the only eigenvalue of \( M \) with modulus \( 1 \) is \( 1 \), and all other eigenvalues satisfy \( |\lambda| < 1 \).
\end{proof}

\newpage

\problem{[Book 6.4.14] Determine whether the following is a natural cubic spline:
\[f(x) = \begin{cases}
    2(x+1)^3 + (x+1)^3 & x \in [-1,0]\\
    3 + 5x + 3x^2 & x \in [0,1]\\
    11 + 11(x-1) + 3(x-1)^2 - (x-1)^3 & x \in [1,2]
\end{cases}\]}
\solution

\begin{proof}
    To determine whether \( f(x) \) is a natural cubic spline on \([-1, 2]\), we need to check the following criteria:
    \begin{enumerate}
        \item \( f(x) \) must be twice continuously differentiable on \([-1, 2] \).
        \item The second derivatives at the endpoints must be zero: \( f''(-1) = f''(2) = 0 \) (natural boundary conditions).
    \end{enumerate}
    \\
    To start with, we simplify each piece of \( f(x) \):
    \\
    1. For \( x \in [-1, 0] \):
    \[
       f(x) = 2(x + 1)^3 + (x + 1)^3 = 3(x + 1)^3.
    \]
    
    2. For \( x \in [0, 1] \):
    \[
       f(x) = 3 + 5x + 3x^2.
    \]
    
    3. For \( x \in [1, 2] \):
       \begin{align*}
            f(x) &= 11 + 11(x - 1) + 3(x - 1)^2 - (x - 1)^3 \\
            &= 11 + 11(x - 1) + 3(x - 1)^2 - (x - 1)^3.
       \end{align*}
    \\
    First, check continuity at the knots \( x = 0 \) and \( x = 1 \):
    \\
    - At \( x = 0 \):
    \[
        \lim_{x \to 0^-} f(x) = 3(0 + 1)^3 = 3(1)^3 = 3,
    \]
    \[
        \lim_{x \to 0^+} f(x) = 3 + 5(0) + 3(0)^2 = 3.
    \]
    So, \( f(x) \) is continuous at \( x = 0 \).
    \\
    - At \( x = 1 \):
    \[
        \lim_{x \to 1^-} f(x) = 3 + 5(1) + 3(1)^2 = 11,
    \]
    \[
        \lim_{x \to 1^+} f(x) = 11 + 11(1 - 1) + 3(1 - 1)^2 - (1 - 1)^3 = 11.
    \]
    So, \( f(x) \) is continuous at \( x = 1 \).
    \\
    Now, Compute the first derivative in each interval:
    \\
    For \( x \in [-1, 0] \):
       \[
       f'(x) = 3 \cdot 3(x + 1)^2 = 9(x + 1)^2.
       \]
    
    2. For \( x \in [0, 1] \):
       \[
       f'(x) = 5 + 6x.
       \]
    
    3. For \( x \in [1, 2] \):
       \[
       f'(x) = 11 + 6(x - 1) - 3(x - 1)^2.
    \]
    \\
    Then, check the continuity of the first derivative at \( x = 0 \):
    \\
    - From the left:
    \[
        \lim_{x \to 0^-} f'(x) = 9(0 + 1)^2 = 9(1)^2 = 9.
    \]
    - From the right:
    \[
        \lim_{x \to 0^+} f'(x) = 5 + 6(0) = 5.
    \]
    Since \( \lim_{x \to 0^-} f'(x) \ne \lim_{x \to 0^+} f'(x) \), the first derivative \( f'(x) \) is not continuous at \( x = 0 \).
    \\
    Conclusion:
    \\
    Because \( f'(x) \) is not continuous at \( x = 0 \), \( f(x) \) is not once continuously differentiable on \([-1, 2] \). A natural cubic spline requires the function to be twice continuously differentiable on the interval and to satisfy the natural boundary conditions. Therefore, \( f(x) \) is \textbf{not} a natural cubic spline.
\end{proof}


\problem{[Book 6.4.25] Determine coefficients \(a,b,c,d\), which make the following a cubic spline:
\[f(x) = \begin{cases}
    x^3 & -1\leq x \leq 0\\
    a+bx+cx^2 + dx^3 & 0\leq x \leq 1
\end{cases}\]}
\solution

\begin{proof}
    Define $f(x)$ and its derivatives:
    \\
    For $x \in [-1, 0]$:
    \[ f(x) = x^3, f^{'}(x) = 3x^2, f^{''}(x) = 6x \]
    \\
    For $x \in [0,1]$:
    \[ f(x) = a + bx + cx^2 + dx^3, f^{'}(x) = b + 2cx + 3dx^2, f^{''}(x) = 2c + 6dx \]
    Continuity at $x = 0$:
    \\
    By definition, function countinuity means: $f(0^{-}) = f(0^{+})$:
    \\
    Calculate each side:
    \[ f(0^{-}) = (0)^3 = 0, f(0^{+}) = a + b(0) + c(0)^2 + d(0)^3 = a \]
    Set them equal so that $a = 0$.
    \\
    By definition, first derivative countinuity countinuity means: $f{'}(0^{-}) = f{'}(0^{+})$:
    Calculate each side:
    \[ f(0^{-}) = 3(0)^2 = 0, f(0^{+}) = b + 2c(0) + 3d(0)^2 = b \]
    Set them equal so that $b = 0$.
    \\
    By definition, second derivative countinuity countinuity means: $f{''}(0^{-}) = f{''}(0^{+})$:
    \\
    Calculate each side:
    \[ f(0^{-}) = 6(0)^3 = 0, f(0^{+}) = 2c + 6d(0) = 2c \]
    Set them equal so that $2c = 0 \Rightarrow c = 0$.
    \\
    Now we need to determine $d$ using the spline's definition:
    \\
    Since $a = b = c = 0$, the functiona for $x \in [0,1]$ simplifies to:
    \[ f(x) = dx^3 \]
    Ensure Smoothness at $x = 1$:
    Eventhough the function is not defined beyond $x = 1$, we typically ant to the spline to be as smooth as possible. 
    In general, $d$ can be any real number.
    \\
    Assuming we want $f(x)$ to be continuous at $x = 1$, and since $f(x) = x^3$ on $[-1, 0]$, it is reasonable to extern this to $[0,1]$ by let $d = 1$.
    \\
    Therefore, the coefficients are:
    \[ a = 0, b = 0, c = 0, d = 1 \]
    This makes $f(x) = x^3$ on both intervals, ensuring that the function and its derivatives are continuous across the entire domain $[-1, 1]$.
\end{proof}

\problem{Let \(f(x) = \arctan(x)\) 
\begin{enumerate}[a)]
    \item Suppose you interpolated \(f(x)\) by a degree \(3\) polynomial using the Chebyshev nodes as \(x\) values [you do not need to calculate the interpolating polynomial]. Estimate the error associated to this interpolation.
    \item Using a taylor series around \(0\), write down a degree \(5\) approximation to \(f(x)\).
    \item With Taylor's form of the remainder, estimate the error associated to the interpolation in (b). (you may use a computer to calculate the 6th derivative, but you must bound it on your own, explaining your work carefully)
    \item Compare your error estimates (a) and (c). Which seems better, and why do you think this might be the case? Hint: taylor series are a little like interpolating just at a single point, using derivatives at just that point to provide extra constraints.
\end{enumerate}
}
\solution
\\
\textbf{(a): }
\begin{proof}
    By definition, for poly. interpolation, the error at a point $x$ is given by: 
    \[ \left| f(x) - P_n(x) \right| = \left| \frac{f^{(n+1)}(\xi)}{(n+1)!} \prod_{i=0}^{n} (x - x_i) \right| \]
    Using the Chebyshev nodes, for $n = 3$ (degree of 3 polynomial), the Chebyshev nodes on the interval $[-1,1]$ are:
    \[ x_k = \cos \left( \frac{2k + 1}{2(n + 1) \pi} \right), \ k = 0,1,2,3 \] 
    Computerd nodes:
    \[ x_0 = \cos(\frac{\pi}{8}) \approx 0.924 \]
    \[ x_1 = \cos(\frac{3\pi}{8}) \approx 0.383 \]
    \[ x_2 = \cos(\frac{5\pi}{8}) \approx -0.383 \]
    \[ x_2 = \cos(\frac{7\pi}{8}) \approx -0.924 \]
    For Chebyshev nodes on $[-1, 1]$, the term $\prod_{i=0}^{n} |x - x_i|$ is bounded by:
    \[ \prod_{i = 0}^{n} |x - x_i| \leq \frac{1}{2^n} \]
    Now we need to find an upper bound $M$ for $|f^{(4)}(x)|$ on $[-1, 1]$.
    \\
    Compute $f^{(4)}(x)$:
    \\
    First derivative:
    \[ f^{(1)}(x) = \frac{1}{1 - x^2} \]
    Second derivative:
    \[ f^{(2)}(x) = \frac{d}{dx} (\frac{1}{1 - x^2}) = - \frac{2x}{(1 + x^2)^2} \]
    Third derivative:
    \[ f^{(3)}(x) = \frac{d}{dx} (- \frac{2x}{(1 + x^2)^2}) = -\frac{2(1 + x^2)^2 - 8x^2(1 + x^2)}{(1 + x^2)^4} = -\frac{2(1 - 3x^2)}{(1 + x^2)^3} \]
    Fourth derivative:
    \[ f^{(4)}(x) = \frac{d}{dx} (-\frac{2(1 - 3x^2)}{(1 + x^2)^3}) = \frac{2(6x(1+x^2)^3 - 3(1-3x^2)(3)(1 + x^2)^2(2x))}{(1 + x^2)^6} \]
    Thus, the absolute value of the fourth derivative is:

    \[
        |f^{(4)}(x)| = \left| \frac{24x(x^2 - 1)}{(1 + x^2)^4} \right|
    \]
    To find an upper bound \( M \) for \( |f^{(4)}(x)| \) on \([-1, 1]\), we analyze the numerator and the denominator separately:

    \begin{itemize}
        \item \textbf{Numerator Analysis:} 
        \\
        - For \( x \in [-1, 1] \), \( |x| \leq 1 \). 
        - Also, \( |x^2 - 1| \leq 1 \), because \( x^2 \leq 1 \), which implies \( |x^2 - 1| \leq 1 \).

        \item \textbf{Denominator Analysis:} 
        \\
        - For \( x \in [-1, 1] \), \( 1 + x^2 \geq 1 \).
        - Therefore, \( (1 + x^2)^4 \geq 1^4 = 1 \).
    \end{itemize}
    \\
    Combining these results, we have:
    \[
        |f^{(4)}(x)| = \left| \frac{24x(x^2 - 1)}{(1 + x^2)^4} \right| \leq \frac{24 \cdot 1 \cdot 1}{1} = 24
    \]
    Thus, the maximum value of \( |f^{(4)}(x)| \) on \( [-1, 1] \) is:
    \[
        M = 24.
    \]
    Lastly, we can apply the error formula:
    \[ |f(x) - P_3(x) \leq \frac{M}{4!} \cdot \prod_{k=0}^{3}|x - x_k| \leq \frac{24}{24} \cdot \frac{1}{8} = \frac{1}{8} = 0.125 \]
    The maximum interpolation error when approximating $f(x) = arctan(x)$ on $[-1,1]$ using a degree 3 polynomial with Chebyshev nodes is bounded by 0.125.
\end{proof}

\textbf{(b): }
\\
\begin{proof}
    Recall that Taylor Series expansion of $arctan(x)$ around $x = 0$ is:
    \[ arctan(x) = \sum_{n = 0}^{\infty} (-1)^n \frac{x^{2n + 1}}{2n + 1}, \ |x| \leq 1 \]
    For Degree of 5 polynomial approximation, we can have:
    \[ f(x) \approx x - \frac{x^3}{3} + \frac{x^5}{5} + O(x^6) \]
\end{proof}

\textbf{(c): }
\\
\begin{proof}
By Taylor Remainder Theorem, the remainder $R_{5}(x)$ for a degree $5$ Taylor polynomial is:
\[ R_5 = \frac{f^{(6)}(\xi)}{6!} \ x^6 \]
By calculator, $f^{(6)}(x)$ is:
\[ f^{(6)}(x) = -\frac{240x(3x^4 -10x^2 + 3)}{(x^2 + 1)^6} \]
To apply the theorem, we need to find an upper bound $M$ $|f^{(6)}(x)|$ on $[-1, 1]$.
\[ |f^{(6)}(x)| = \left|-\frac{240x(3x^4 -10x^2 + 3)}{(x^2 + 1)^6} \right| = \frac{240|x||3x^4 -10x^2 + 3|}{(x^2 + 1)^6} \]
Bounding $|x|$: 
\[ |x| \leq 1 \ for \ x \in [-1, 1] \]
Bounding $|3x^4 -10x^2 + 3|$: 
\\
Let $g(x) = 3x^4 - 10x^2 + 3$. To find the maximum of $|g(x)|$ on $[-1,1]$, we analyze its critical points and endpoints.
\\
1. Find critical points:
\[ g'(x) = 12x^3 - 20x = 4x(3x^2 - 5) = 0 \]
\[ \Rightarrow x = 0 \ or \ x = \pm \sqrt{\frac{5}{3}} \approx \pm 1.291 \]
Only $x = 0$ lies within $[-1, 1]$.
\\
2. Evaluate $g(x)$ at critical and end points:
\[ g(0) = 3(0)^4 - 10(0)^2 + 3 = 3 \]
\[ g(1) = 3(1)^4 - 10(1)^2 + 3 = 3 - 10 + 3 = -4 \Rightarrow |g(1)| = 4 \]
\[ g(-1) = 3(-1)^4 - 10(-1)^2 + 3 = 3 - 10 + 3 = -4 \Rightarrow |g(1)| = 4 \]
Thus, \( |g(x)| \leq 4 \ for \ x \in [-1, 1] \)
\\
Bounding the denominator $(x^2 + 1)^6$:
On \([-1,1]\):
\[
1 \leq x^2 + 1 \leq 2 \quad \Rightarrow \quad 1^6 \leq (x^2 + 1)^6 \leq 2^6 = 64
\]
Lastly, we combine the bounds:
\[
|f^{(6)}(x)| = \frac{240|x||3x^4 -10x^2 + 3|}{(x^2 + 1)^6} \leq \frac{240 \times 1 \times 4}{1} = 960 \quad \text{at } x = \pm1
\]
However, evaluating at \( x = \pm1 \):
\[
|f^{(6)}(\pm1)| = \frac{240 \times 1 \times 4}{(1 + 1)^6} = \frac{960}{64} = 15
\]
Since \( |f^{(6)}(x)| \) attains its maximum at \( x = \pm1 \), we set:
\[
M = 15
\]
Substituting \( M = 15 \) and \( 6! = 720 \) into the remainder formula:
\[
|R_5(x)| \leq \frac{15}{720} |x|^6 = \frac{1}{48} |x|^6 \approx 0.0208 |x|^6
\]
Since \( |x| \leq 1 \) on \([-1,1]\):
\[
    |R_5(x)| \leq \frac{1}{48} \approx 0.0208
\]
\end{proof}

\textbf{(d): }
\\
In comparing the error estimates from parts (a) and (c), the Taylor series approximation (part~c) yields a significantly smaller error bound of \( |R_5(x)| \leq 0.0208 \) over the interval \([-1, 1]\), 
compared to the Chebyshev interpolation (part a) which has an error bound of \( |f(x) - P_3(x)| \leq 0.125 \). 
This superior accuracy of the Taylor approximation arises because it utilizes a higher-degree polynomial (degree~5 versus degree~3) and incorporates derivative information at a single point (\(x=0\)), allowing for a more precise local fit. 
In contrast, Chebyshev interpolation distributes interpolation nodes across the entire interval to minimize the maximum error uniformly but does not exploit derivative information, resulting in a larger overall error bound. 
Therefore, the Taylor series provides a better error estimate in this case due to its enhanced local accuracy near the expansion point.

\problem{Determine a quadratic spline approximation \(S(x)\) to \(f(x) = \arctan(x)\) with nodes \(-1,0,1\).}
\solution
\begin{proof}
    First we need to define $S(x)$ as a peicewise quadratic function:
    \[ S(x)
        \begin{cases}
            S_1(x) = a_1x^2 + b_1x + c_1, \ for \ x \in [-1, 0], \\
            S_2(x) = a_2x^2 + b_2x + c_2, \ for \ x \in [0, 1], \\
        \end{cases}
    \]
    Then we apply the interpolation conditions, computing the function vlaues at the nodes:
    \[ f(-1) = arctan(-1) = -\frac{\pi}{4} \]
    \[ f(0) = arctan(0) = 0 \]
    \[ f(1) = arctan(1) = \frac{\pi}{4} \]
    interpolation at $x = -1$:
    \[ S_1(-1) = a_1(-1)^2 + b_1(-1) + c_1 = -\frac{\pi}{4} \] 
    interpolation at $x = 0$:
    For both $S_1(x)$ and $S_2(x)$: 
    \[ S_1(0) = c_1 = 0, \ S_2(0) = c_2 = 0 \]
    interpolation at $x = 1$:
    \[ S_2(-1) = a_2(1)^2 + b_2(1) + c_2 = \frac{\pi}{4} \] 
    Then, we apply continuity conditions at $x = 0$,
    \\
    Continuity of the function at $x = 0$:
    \[ S_1(0) = S_2(0) \Longrightarrow c_1 = c_2 = 0 \]
    Continuity of the First Derivative at $x = 0$:
    \\
    \[ S_1'(x) = 2a_1x + b_1 \]
    \[ S_2'(x) = 2a_2x + b_2 \]
    At $x = 0$:
    \[ S_1'(x) = b_1, \ S_2'(x) = b_2 \]
    Set them equal:
    \[ b_1 = b_2 = b \]
    After that, we apply second continuity conditions at endpoints,
    \\
    Compute the function's second derivatives at the endpoints:
    \[ f'(x) = \frac{1}{1 + x^2} \]
    \[ f''(x) = -\frac{2x}{(1 + x^2)^2} \]
    At $x = -1$:
    \[ f''(-1) = -\frac{2(-1)}{(1 + (-1)^2)^2} = \frac{2}{4} \frac{1}{2} \]
    Set:
    \[ S_1''(x) = 2a_1 = f''(-1) \Longrightarrow a_1 = \frac{1}{4} \]
    At $x = 1$:
    \[ f''(1) = - \frac{2(1)}{(1 + 1^2)^2} = - \frac{2}{4} = - \frac{1}{2} \]
    Set:
    \[  S_2''(x) = 2a_2 = f''(1) \Longrightarrow a_2 = - \frac{1}{4} \]
    Now we can solve for the remaining coefficients,
    \\
    Equation from $x = -1$:
    \[ S_1(-1) = a_1(-1)^2 + b_1(-1) + c_1 = \frac{1}{4}(1) - b_1 + 0 = -\frac{\pi}{4} \]
    \[ \Rightarrow \frac{1}{4} - b_1 = -\frac{\pi}{4} \Longrightarrow b_1 = \frac{1}{4} + \frac{\pi}{4} = \frac{1 + \pi}{4} \]
    Equation from $x = 1$:
    \[ S_2(-1) = a_2(-1)^2 + b_2(-1) + c_2 = \frac{1}{4}(1) - b_2 + 0 = \frac{\pi}{4} \]
    \[ \Rightarrow -\frac{1}{4} - b_2 = \frac{\pi}{4} \Longrightarrow b_2 = \frac{1}{4} + \frac{\pi}{4} = \frac{1 + \pi}{4} \]
    Since $b_1 = b_2 = b$, this is consistent.
    \\
    Now, we can Substituting the coefficients into $S_1(x)$ and $S_2(x)$:
    For $x \in [-1, 0]$:
    \[ S_1(x) = a_1x^2 + bx + c_1 = \frac{1}{4}x^2 + \frac{1 + \pi}{4}x + 0 \]
    \[ S_2(x) = a_2x^2 + bx + c_2 = -\frac{1}{4}x^2 + \frac{1 + \pi}{4}x + 0 \]
    Lastly, to ensure the validity of the approximation, we need to do some verification:
    \\
    At $x = -1$:
    \[ S_1(-1) = \frac{1}{4}(-1)^2 + \frac{1 + \pi}{4}(-1) = \frac{1}{4} - \frac{1 + \pi}{4} = -\frac{\pi}{4} \]
    which matches $f(-1) = -\frac{\pi}{4}$
    \\
    At $x = 0$:
    \[ S_1(0) = S_2(0) = 0 \]
    Matches $f(0) = 0$
    \\
    At $x = 1$:
    \[ S_2(1) = -\frac{1}{4}(1)^2 + \frac{1 + \pi}{4}(1) = -\frac{1}{4} + \frac{1 + \pi}{4} = \frac{\pi}{4} \]
    Matches $f(1) = \frac{\pi}{4}$
    \\
    And for checking the first derivative continuity at $x = 0$;:
    \[ S'_1(x) = 2a_1x + b = \frac{1}{2}x + \frac{1 + \pi}{4}, \ S'_1(0) = \frac{1 + \pi}{4} \]
    \[ S'_2(x) = 2a_2x + b = -\frac{1}{2}x + \frac{1 + \pi}{4}, \ S'_2(0) = \frac{1 + \pi}{4} \]
    $ S'_1(x) = S'_2(x) $, which ensures the validity.
    \\
    The quadratic spline approximation $S(x)$ is:
    \[ 
        S(x) = 
        \begin{cases} 
        \frac{1}{4}x^2 + \frac{1+\pi}{4}x, & \text{for } x \in [-1, 0], \\ 
        -\frac{1}{4}x^2 + \frac{1+\pi}{4}x, & \text{for } x \in [0, 1]. 
        \end{cases}
    \]
\end{proof}

\problem{Let \(f(x) = 4x^2 - 4^{x}\).
\begin{enumerate}
    \item Using the intermediate value theorem, show that \(f(x)\) has at least one root in \([-1,0]\) and another in \([0,1.5]\).
    \item Interpolate \(f(x)\) by a degree \(3\) polynomial using nodes \(x=-1/2,0,1/2\).
    \item Use the interpolation to estimate the roots of \(f(x)\) in those intervals.
\end{enumerate}}
\solution

\textbf{(1): }
\\
\begin{proof}
    For interval $[-1, 0]$, compute $f(-1)$ and $f(0)$.
    \\
    At $x = -1$:
    \[ f(-1) = 4(-1)^2 - 4^{-1} = 4(1) - \frac{1}{4} = 4 - \frac{1}{4} = \frac{15}{4} > 0 \]
    At $x = 0$:
    \[ f(0) = 4(0)^2 - 4^{0} = 0 - 1 = -1 < 0 \]
    Since $f(-1) > 0$ and $f(0) < 0$, and $f(x)$ is continuous on $[-1, 0]$, by IVT, there is at least one root in $[-1, 0]$.
    \\
    For interval $[0, 1.5]$, compute $f(0)$ and $f(1.5)$.
    \\
    At $x = 0$:
    \[ f(0) = -1 < 0 \]
    At $x = 1.5$:
    \[ f(1.5) = 4(1.5)^2 - 4^{1.5} = 4(2.25) - 4^{1.5} = 9 - 4^{1.5} = 9 - 8 = 1 > 0 \]
    Since $f(0) < 0$ and $f(1.5) > 0$, and $f(x)$ is continuous on $[0, 1.5]$, by IVT, there is at least one root in $[0, 1.5]$.
\end{proof}

\textbf{(2): }
\begin{proof}
As an additional condition for interpolation, we use the derivative at \( x = 0 \):
\[ f'(0) = 8x - 4^x \ln(4) = -\ln 4 \]
Now, we need to compute \( f(x) \) at the nodes:
\[ f\left( -\dfrac{1}{2} \right) = 1 - \dfrac{1}{2} = \dfrac{1}{2} \]
\[ f(0) = -1 \]
\[ f\left( \dfrac{1}{2} \right) = 1 - 2 = -1 \]
After that, we can set up the interpolation conditions:
\[ P\left( -\dfrac{1}{2} \right) = \dfrac{1}{2}, \ P(0) = -1 \]
\[ P\left( \dfrac{1}{2} \right) = -1, P'(0) = -\ln 4 \]
Now we can write the equations to solve for coefficients
\[ -\dfrac{a}{8} + \dfrac{b}{4} - \dfrac{c}{2} + d = \dfrac{1}{2} \].
\[ \dfrac{a}{8} + \dfrac{b}{4} + \dfrac{c}{2} -1 = -1 \Rightarrow \dfrac{a}{8} + \dfrac{b}{4} + \dfrac{c}{2} = 0 \]
\[ d = -1, c = -\ln 4 \]
Simplify equations:
\\
Equation (1):
\[
-\dfrac{a}{8} + \dfrac{b}{4} + \dfrac{\ln 4}{2} -1 = \dfrac{1}{2}.
\]
Multiply both sides by 8:
\[
- a + 2b + 4 \ln 4 -8 = 4.
\]
Simplify:
\[
- a + 2b + 4 \ln 4 = 12. \quad (1a)
\]
Equation (3):
\[
\dfrac{a}{8} + \dfrac{b}{4} - \dfrac{\ln 4}{2} = 0.
\]
Multiply both sides by 8:
\[
a + 2b - 4 \ln 4 = 0. \quad (3a)
\]
Then, we can solve for \( a \) and \( b \):
\\
Add equations (1a) and (3a):
\[
(-a + a) + (2b + 2b) + (4 \ln 4 - 4 \ln 4) = 12 + 0 \implies 4b = 12 \implies b = 3.
\]
Substitute \( b = 3 \) into (3a):
\[
a + 2(3) - 4 \ln 4 = 0 \implies a = 4 \ln 4 -6.
\]
Since \( \ln 4 = 2 \ln 2 \), we have:
\[
a = 8 \ln 2 -6, \quad c = -2 \ln 2, \quad d = -1.
\]
Lastly, the Final Interpolating Polynomial will be
\[
    P(x) = (8 \ln 2 -6) x^3 + 3 x^2 - 2 \ln 2 \, x -1.
\]
\end{proof}

\textbf{(3): }
\\
From Part 2, the interpolating polynomial is:
\[
P(x) = (8 \ln 2 - 6) x^3 + 3 x^2 - 2 \ln 2 \, x - 1.
\]
Using \( \ln 2 \approx 0.6931 \), the polynomial becomes:
\[
P(x) = -0.4552\, x^3 + 3 x^2 - 1.3862\, x - 1.
\]
Estimating the Root in \([-1, 0]\)
\\
Evaluate \( P(x) \) at the endpoints:
\[
\begin{cases}
P(-1) = -0.4552(-1)^3 + 3(-1)^2 - 1.3862(-1) - 1 \approx 3.8414 > 0, \\
P(0) = -1 < 0.
\end{cases}
\]
Since \( P(-1) > 0 \) and \( P(0) < 0 \), by the Intermediate Value Theorem, there is a root \( x_1 \) in \([-1, 0]\).
Using the Bisection Method:
\begin{enumerate}
    \item First midpoint: \( x_{\text{mid}} = -0.5 \Rightarrow P(-0.5) \approx 0.5 > 0 \Rightarrow \) root is in \([-0.5, 0]\).
    \item Second midpoint: \( x_{\text{mid}} = -0.25 \Rightarrow P(-0.25) \approx -0.46 < 0 \Rightarrow \) root is in \([-0.5, -0.25]\).
    \item Third midpoint: \( x_{\text{mid}} = -0.375 \Rightarrow P(-0.375) \approx -0.034 < 0 \Rightarrow \) root is in \([-0.5, -0.375]\).
    \item Fourth midpoint: \( x_{\text{mid}} = -0.4375 \Rightarrow P(-0.4375) \approx 0.22 > 0 \Rightarrow \) root is in \([-0.4375, -0.375]\).
\end{enumerate}
By continuing this process, we estimate:
\[
    x_1 \approx -0.41.
\]
Estimating the Root in \([0, 1.5]\):
\\
Evaluate \( P(x) \) at \( x = 0 \) and \( x = 1 \):
\[
\begin{cases}
P(0) = -1 < 0, \\
P(1) \approx 0.1586 > 0.
\end{cases}
\]
There is a root \( x_2 \) in \([0, 1]\).
\\
Using the Bisection Method:
\begin{enumerate}
    \item First midpoint: \( x_{\text{mid}} = 0.5 \Rightarrow P(0.5) \approx -1 < 0 \Rightarrow \) root is in \([0.5, 1]\).
    \item Second midpoint: \( x_{\text{mid}} = 0.75 \Rightarrow P(0.75) \approx -0.544 < 0 \Rightarrow \) root is in \([0.75, 1]\).
    \item Third midpoint: \( x_{\text{mid}} = 0.875 \Rightarrow P(0.875) \approx -0.151 < 0 \Rightarrow \) root is in \([0.875, 1]\).
    \item Fourth midpoint: \( x_{\text{mid}} = 0.9375 \Rightarrow P(0.9375) \approx 0.0025 > 0 \Rightarrow \) root is in \([0.875, 0.9375]\).
\end{enumerate}
\\
By continuing this process, we estimate:
\[
x_2 \approx 0.93.
\]
Conclusion:
\\
Using the interpolating polynomial \( P(x) \), we estimate the roots of \( f(x) \) in the specified intervals:
\[
\begin{cases}
\text{Root in } [-1, 0]: & x_1 \approx -0.41, \\
\text{Root in } [0, 1.5]: & x_2 \approx 0.93.
\end{cases}
\]

\end{document}
