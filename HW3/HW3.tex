\documentclass{article}
\usepackage[margin=1in]{geometry}
\usepackage{amsmath, amsthm, amssymb, amsfonts}
\usepackage{graphicx}
\usepackage{xcolor}
\usepackage{enumerate}

\newcommand{\inv}{^{-1}}

\newcounter{pnum}
\newcommand\problem[1]{\stepcounter{pnum}\begin{center}\fbox{\begin{minipage}{0.95\linewidth}
\textbf{\thepnum.} #1
\end{minipage}}\end{center}}

\newcommand\solution{\vspace{0.4em}\textbf{Solution.}}

\title{Homework 3 (error and iterative methods, markov chains)}
\author{Hanzhang Yin}
\begin{document}

\maketitle

Recall that a Markov matrix is an \(n\times n\) matrix whose \textit{columns} represent the probabilities of transitioning between \(n\) states: the columns sum to \(1\) and entry \(M_{ij}\) is the probability of transitioning from state \(j\) to state \(i\). Some sources use the transpose of this matrix.

\problem{Apply iterative methods (our generalized Jacobi method) to estimate a solution to
    \[\begin{bmatrix}
        4 & 5 \\ 3 & 5
    \end{bmatrix}
    x =
    \begin{bmatrix}
       2\\3 
    \end{bmatrix}\]

     \begin{enumerate}[a)]
         \item Pick any splitting matrix besides \(A\inv\), and any initial guess besides the actual solution, and determine the result after two iterations.
         \item Determine the quantity \(\delta = ||I-Q\inv A||\) (notation from class/book; \(Q\) is the splitting matrix).
         \item Determine the actual solution by inverting the matrix.
         \item Compare the actual solution to your approximate solution from (a) using the \(\infty\) norm. Then, compare to the error estimate theorem from class. What do you notice?
     \end{enumerate}
     }
\solution
\\
\textbf{(a): }
\\
We pick splitting matrix \( Q = \begin{bmatrix} 4 & 0 \\ 0 & 5 \end{bmatrix} \), as the diagonal part of $A$. The Remainder matrix $R$ will be:
\[ R = Q - A = \begin{bmatrix} 4 & 0 \\ 0 & 5 \end{bmatrix} - \begin{bmatrix} 4 & 5 \\ 3 & 5 \end{bmatrix} = \begin{bmatrix} 0 & -5 \\ -3 & 0 \end{bmatrix} \]
Selectg initial guess \( x^{0} = \begin{bmatrix} 0 \\ 0 \end{bmatrix} \).
\\
Using Cramer's rule the inverse of $Q$ is:
\[ Q^{-1} = \frac{1}{det(Q)} Q = \frac{1}{20} \begin{bmatrix} 4 & 0 \\ 0 & 5 \end{bmatrix} = \begin{bmatrix} \frac{1}{4} & 0 \\ 0 & \frac{1}{5} \end{bmatrix} \]
Denote matrix \( M = Q^{-1}R \), we can get:
\[ M = Q^{-1}R = \begin{bmatrix} \frac{1}{4} & 0 \\ 0 & \frac{1}{5} \end{bmatrix} \begin{bmatrix} 0 & -5 \\ -3 & 0 \end{bmatrix} = \begin{bmatrix} 0 & -\frac{5}{4} \\ -\frac{3}{5} & 0 \end{bmatrix} \]
Denote vector \( c = Q^{-1}b \), we can get:
\[ c = \begin{bmatrix} \frac{1}{4} & 0 \\ 0 & \frac{1}{5} \end{bmatrix} \begin{bmatrix} 2 \\ 3 \end{bmatrix} = \begin{bmatrix} \frac{1}{2} \\ \frac{3}{5} \end{bmatrix} \]
Now, we can denote the first iteration (k = 0) as:
\[ x^{(1)} = Mx^{(0)} + c = \begin{bmatrix} \frac{1}{2} \\ \frac{3}{5} \end{bmatrix} \]
Then second iteration (K = 1):
\[ x^{(1)} = Mx^{(1)} + c = \begin{bmatrix} 0 & -\frac{5}{4} \\ -\frac{3}{5} & 0 \end{bmatrix} \begin{bmatrix} \frac{1}{2} \\ \frac{3}{5} \end{bmatrix} + \begin{bmatrix} \frac{1}{2} \\ \frac{3}{5} \end{bmatrix} \]
\[
    = \begin{bmatrix} (-\frac{5}{4}) \cdot \frac{3}{5} \\ (-\frac{3}{5}) \cdot \frac{1}{2} \end{bmatrix} + \begin{bmatrix} \frac{1}{2} \\ \frac{3}{5} \end{bmatrix} 
    = \begin{bmatrix} -\frac{3}{4} + \frac{1}{2} \\ -\frac{3}{10} + \frac{3}{5} \end{bmatrix}
    = \begin{bmatrix} -\frac{1}{4} \\ \frac{3}{10} \end{bmatrix} 
\]
\textbf{(b): }
\\
Compute \( Q^{-1}A \):
\[ Q^{-1}A = \begin{bmatrix} \frac{1}{4} & 0 \\ 0 & \frac{1}{5} \end{bmatrix} \begin{bmatrix} 4 & 5 \\ 3 & 5 \end{bmatrix} = \begin{bmatrix} 1 & \frac{5}{4} \\ \frac{3}{5} & 1 \end{bmatrix} \]
Then \( I - Q^{-1}A \) will be:
\[ I - Q^{-1}A = \begin{bmatrix} 1 & 0 \\ 0 & 1 \end{bmatrix} - \begin{bmatrix} 1 & \frac{5}{4} \\ \frac{3}{5} & 1 \end{bmatrix} = \begin{bmatrix} 0 & -\frac{5}{4} \\ -\frac{3}{5} & 0 \end{bmatrix} \]
Using Infinity Norm to compute $\delta$:
\[ \delta = || I - Q^{-1}A ||_{\infty} = \max \left( |0| + |-\frac{5}{4}|, |-\frac{3}{5}| + |0| \right) = \max \left( \frac{5}{4}, \frac{3}{5} \right) = \frac{5}{4} \]
\textbf{(c): }
\\
\[ \det(A) = (4)(5) - (3)(5) = 20 - 15 = 5 \]
\[ A^{-1} = \frac{1}{5} \begin{bmatrix} 5 & -5 \\ -3 & 4 \end{bmatrix} = \begin{bmatrix} 1 & -1 \\ -\frac{3}{5} & \frac{4}{5} \end{bmatrix} \]
Then the exact solution $x$ will be:
\[ x = A^{-1}b = \begin{bmatrix} 1 & -1 \\ -\frac{3}{5} & \frac{4}{5} \end{bmatrix} \begin{bmatrix} 2 \\ 3 \end{bmatrix} = \begin{bmatrix} -1 \\ 1.2 \end{bmatrix} \]
\textbf{(d): }
The error after two iteration can be calculated as:
\[ e^{(2)} = x - x^{(2)} = \begin{bmatrix} -1 \\ 1.2 \end{bmatrix} - \begin{bmatrix} -\frac{1}{4} \\ \frac{3}{10} \end{bmatrix} = \begin{bmatrix} -\frac{3}{4} \\ 0.9 \end{bmatrix} \]
Then,
\[ ||e^{(2)}||_{\infty} = \max \left(|-\frac{3}{4}|, |0.9| \right) = 0.9 \]
The error estimate theorem states:
\[ ||e^{(k)}|| \leq \delta^k ||e^{(0)}|| \]
In here, $\delta = \frac{5}{4}$ and $||e^{(0)}||_{\infty} = \max(|-1|, |1.2|) = 1.2$
\\
The theoretical error bound after 2 iter. hence will be:
\[ \delta^k ||e^{(0)}|| = \left( \frac{5}{4} \right) \times 1.2 = \frac{25}{16} \times 1.2 \approx 1.875 \]
Noting that actual error: $||e^{(2)}||_{\infty} = 0.9$, theoretical error bound: $||e^{(2)}||_{\infty} \leq 1.875$. 
\\
Although the error bound suggests divergence for \( \delta > 1 \), the error decreases initially before increasing as iterations progress, confirming divergence as predicted and consistent with the error estimate theorem.


\problem{In our steady-state calculation for a Markov matrix \(M\), we determined that all Markov matrices have \(1\) as an eigenvalue by iterative methods. Iterative methods requires some technical assumptions that we did not discuss. This problem walks you through verifying this fact without without them.
\begin{enumerate}[a)]
    \item Show that a Markov matrix \(M\) has \(1\) as a left eigenvalue (i.e. an eigenvalue of \(M^T\)). Hint: the sum of the rows is \(1\) in a Markov matrix -- what left vector multiplication would produce such a row sum? Is it an eigenvector?
    \item Show that \(A\) and \(A^T\) have the same minimal polynomial. Hint: check that \(p(A)^T = p(A^T)\) for any polynomial \(p\).
    \item Combine the previous two facts to conclude that \(A\) has an eigenvector with eigenvalue \(1\).
\end{enumerate}
}
\solution
\\
\textbf{(a): }
\\
\begin{proof}
    let $v$ be a $n \times 1$ vector s.t. $v = \begin{bmatrix} 1 \\ 1 \\ \vdots \\ 1 \end{bmatrix}$, by the hint, since Markov matrix $M$ is row-stochastic (i.e. rows sum to 1), we can have:
    \[ (M \cdot v)_i = \sum_{j = 1}^{n} M_{ij} \cdot v = \sum_{j = 1}^{n} M_{ij} = 1 \]
    Therefore,
    \[ Mv = \begin{bmatrix} 1 \\ 1 \\ \vdots \\ 1 \end{bmatrix} = v \]
    Then we can transpose to both sides:
    \[ (Mv)^T = v^T \Rightarrow v^T M^T = v^T \]
    Hence, by definition, $v^T$ is a left eigenvector of $M^T$ corresponding to eigenvalue $\lambda = 1$.
\end{proof}

\textbf{(b): }
\\
by the given hint, we will show that for any matrix $A$, \( p(A)^T = p(A^T) \) holds and analyze from here.
\begin{proof}
    Note, for any positive integer $k$, by applying property \((AB)^T = B^TA^T\) recusively, \( (A^k)^T = (A^T)^k \) holds.
    \\
    Let $p(x)$ by any polynomial, we can denote it as \( p(x) = \sum_{i=0}^{n}c_i x^i\), then:
    \[ p(A)^T = \left( \sum_{i=0}^{n}c_i A^i \right)^T = \sum_{i=0}^{n}c_i (A^i)^T = \sum_{i=0}^{n}c_i (A^T)^i = p(A^T) \]
    \\
    Let $m_A(x)$ be the minial polynomial of $A$, so $M_A(A) = 0$, then:
    \[ m_A(A^T) = m_A(A)^T = 0^T = 0 \]
    Similarly, if $M_{A^T}(A)$ is the minimal polynomial for $A^T$, then:
    \[ m_{A^T}(A) = m_{A^T}(A^T)^T = 0^T = 0 \]
    Noting that, $m_A(x)$ annihilates $A$ and $m_{A^T}(A)$ annihilates $A^T$, and minial polynomials are unique monic polynomials of least degree:
    \[ m_{A^T}(x) = m_{A}(x) \]
\end{proof}

\textbf{(c): }
\begin{proof}
    From part (a), denote $M = A$, we get that $A^T$ has an eigenvalue $\lambda = 1$.
    \\
    Since $A^T$ has an eigenvalue 1, its minimal polynomial $m_{A^T}(x)$ contains factor $(x - 1)$
    \\
    And, from part (b), we know $m_{A^T}(x) = m_{A}(x)$. Therefore $m_{A}(x)$ must aslo contains factor $(x - 1)$. 
    This means $A$ has an eigenvalue $\lambda = 1$
    \\
    By definition of eigenvalue, there exists a non-zsero vector $v$ s.t. $Av = \lambda v = v$.
    \\
    Therefore, $A$ has eigenvalue $1$, with corresponding eigenvector.
\end{proof}


\problem{Recall that we assumed a matrix has a full-rank eigenspace and a unique largest eigenvalue in order to locate its largest eigenvalue (and associated eigenvector) by iterative methods.

 We saw above that every Markov matrix \(M\) has \(1\) as an eigenvalue. 
\begin{enumerate}
    \item Prove every eigenvalue of \(M\) has norm at most \(1\). Hint: use (2b) and the \(\infty\) norm, or, equivalently, the Gershgorin circle theorem.
    \item Suppose that \(M\) is \(2\times 2\) with full rank eigenspace, but \(1\) is a repeated eigenvalue. What does this mean for the original  matrix?
    \item Suppose that \(M\) is \(2\times 2\) and has \(1\) as its only eigenvalue, but its eigenspace is not full rank. What can you say about this situation? Hint: use the \(\infty\)-norm and examine the left hand side of \(|Mv| \leq ||M|||v| = |v|\) more carefully.
\end{enumerate}}
\solution
\\
\textbf{(a): }
\begin{proof}
    Let $M$ by an $n \times n$ Markov matrix. The $\infty$-norm of $M$ is:
    \[ ||M||_{\infty} = \max_{1 \leq i \leq n} \sum_{j = 1}^{n} |M_{ij}| = 1 \]
    By the submultiplicative property of matrix norms and $M$ is non-negative matrix,
    \[ \rho(M) \leq ||M||_{\infty} = 1 \]
    where $\rho(M)$ denotes the spectral radius of $M$. Hence eveyr eigenvalue $\lambda$ of $M$ suffices $|\lambda| \leq 1$.
\end{proof}

\textbf{(b): }
\begin{proof}
    Let $M$ be a $2 \times 2$ Markov matrix with eigenvalue $\lambda = 1$ of both algebraic multiplicity and geometric multiplicity $2$ (since $M$ is full rank). 
    When algebraic multiplicity = geometric multiplicity, $M$ is diagonalizable. Hence,
    \[ M = P D P^{-1} = I \]
    Where $D$ is the diagonal matrix with egienvlaues on its diagonal, and $P$ is eigenvectors.
    Given both eigenvalues are $1$, $D = I$, the identity matrix, hence:
    \[ M = P \begin{bmatrix} 1 & 0 \\ 0 & 1 \end{bmatrix} P^{-1} = I \]
    If a $2 \times 2$ Markov matrix $M$ has a full rank eigenspace with $1$ as a repeated eigenvlaue, then $M$ must be the identity matrix.
\end{proof}

\textbf{(c): }
\begin{proof}
    Let $M$ be a $2 \times 2$ Markov matrix with eigenvalue $\lambda = 1$ of both algebraic multiplicity $2$ but geometric multiplicity $1$ (since $M$ is NOT full rank). 
    Then, $M$ is defective and cannot be diagonalized. Its Jordan form by definition, will be:
    \[ J = \begin{bmatrix} 1 & 1 \\ 0 & 1 \end{bmatrix} \]
    so, \( M = PJP^{-1} \) for some invertible $P$ matrix.
    \\
    Now consider the iterative behavior of $M$:
    \[ M^{k} = P J^{k} P^{-1} = P \begin{bmatrix} 1 & k \\ 0 & 1 \end{bmatrix} P^{-1} \]
    As $k \rightarrow \infty$, the off-diagonal entry $k$ causes $M^k$ to grow without bound.
    \\
    Then, we can compute the $\infty$-norm of $M$:
    \[ ||M||_{\infty} = \max_{1 \leq i \leq n} \sum_{j = 1}^{n} |M_{ij}| = 1 \]
    by the hint, for any vector $v \in \mathbb{R}^2$, supposingly:
    \[ |Mv|_{\infty} \leq ||M||_{\infty} |v|_{\infty} = |v|_{\infty} \]
    But, from the iterative behavior:
    \[ |Mv|_{\infty} \leq |v|_{\infty} \]
    yet $M^k$ grows without bound unless $v$ is a eigenvector.
    \\
    Now, let $v$ be a generalized eigenvector s.t. $(M - I)v = w$, here $w$ is an eigenvector. Then:
    \[ Mv = v + w \]
    Applying the norm:
    \[ |Mv|_{\infty} = |v + w|_{\infty} \leq |v|_{\infty} \]
    Since $w$ is non-zero and not a scalar multiple of $v$ (since $M$ is defective), adding $w$ to $v$ violating the inequality.
    Hence, we derived a contradiction.
    \\
    By proof by contradiction, no such defective $2 \times 2$ Markov matrix $M$ exists.
\end{proof}

\problem{(bonus) Generalize (3c) to show that a Markov matrix with no zeros has a \textit{unique} eigenvector whose associated eigenvalue has norm \(1\).}
\solution


\problem{(ungraded bonus) If you know some graph theory, discuss the implications for a Markov matrix whose state diagram, like the weather one we drew in class, is (directed) connected and such that each state has a nonzero probability of remaining the same. Hint: could the fact above apply to a large power of \(M\)?}
\solution

\end{document}
