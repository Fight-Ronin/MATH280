\documentclass[12pt]{article}
\usepackage{amsmath, amsthm, amssymb, amsfonts}
\usepackage[margin=1in]{geometry}
\usepackage[backend=biber]{biblatex}
\bibliography{app_log.bib}

\newcounter{entry}
\newcommand\ALEntry[1]{\stepcounter{entry}\textbf{[\theentry]} \fullcite{#1}.\\[-0.2em]}

\title{\vspace{-2cm}Project 2 Pre-report}
\author{Hanzhang Yin}

\begin{document}
\maketitle

\noindent This project evaluates the effectiveness and challenges of applying the bisection method, Newton's method, and the secant method to selected polynomial and non-polynomial equations. By analyzing the strengths and weaknesses of each method, we predict their performance and potential issues, highlighting possible limitations and areas where convergence may be difficult.

\subsection*{Univariate Polynomial Equations}

\noindent \textbf{Polynomial 1: Lagrange Points in Orbital Mechanics}

\noindent The quintic polynomial equation for the \( L_1 \) Lagrange point presents challenges due to the potential for closely spaced roots when \( \mu \) approaches zero. This condition may lead to convergence issues, especially for Newton’s and secant methods, as small derivatives near closely spaced roots can cause instability.

\noindent \textbf{Expectations:}
\begin{itemize}
    \item \textbf{Bisection Method}: Reliable within intervals with sign changes, but likely to converge slowly due to the complex root distribution.
    \item \textbf{Newton’s Method}: Expected to perform efficiently with good initial guesses, but may struggle near multiple or closely spaced roots, potentially causing divergence or slow convergence.
    \item \textbf{Secant Method}: Similar to Newton's, effective with close initial estimates but may face challenges with multiple roots due to the lack of guaranteed quadratic convergence.
\end{itemize}

\noindent \textbf{Predictions:}  
Newton’s and secant methods may face convergence difficulties in degenerate cases (e.g., \( \mu \rightarrow 0 \)), where closely spaced roots increase sensitivity. The bisection method, though slower, should manage to find a root with an appropriate interval.

\noindent \textbf{Polynomial 2: Ideal Gas Law for Real Gases}

\noindent The cubic polynomial from the modified ideal gas law exhibits root convergence near the critical point, posing challenges for Newton's and secant methods. This sensitivity around the critical point could lead to erratic error patterns due to rapidly changing derivatives.

\noindent \textbf{Expectations:}
\begin{itemize}
    \item \textbf{Bisection Method}: Effective for simpler cases, though likely slower near closely spaced roots at critical points.
    \item \textbf{Newton’s Method}: Performs well with an accurate initial guess but may struggle with small derivatives near critical conditions, leading to potential instability.
    \item \textbf{Secant Method}: Expected to show similar results to Newton's, though potentially with slower convergence. It may require careful starting points to avoid instability near the critical point.
\end{itemize}

\noindent \textbf{Predictions:}  
Newton’s and secant methods might need careful parameter handling around critical points, where convergence can be erratic. Bisection should reliably identify a root if the interval is chosen based on sign changes.

\subsection*{System of Polynomial Equations: Coupled Spring-Mass System}

The coupled spring-mass system may cause numerical issues when parameters, such as \( k_2 \), are small, leading to nearly singular matrices. This can impact the stability of multivariable Newton’s method, as the system’s sensitivity increases under such conditions.

\noindent \textbf{Expectations:}
\begin{itemize}
    \item \textbf{Multivariable Newton’s Method}: Effective for solving the system, though it may encounter stability issues with nearly singular Jacobians, particularly when parameters approach degenerate values.
\end{itemize}

\noindent \textbf{Predictions:}  
Convergence may be challenging if the system is close to singular, impacting Jacobian inversion. Careful initial guesses and analysis of Jacobian conditioning are critical for stability.

\subsection*{Non-Polynomial Equation: Sinusoidal Motion in Spring Displacement}

The equation \( 5 \sin(2t) - 3 = 0 \) for sinusoidal spring motion presents an oscillatory behavior, potentially causing convergence issues for iterative methods. Oscillatory functions are challenging, as they may lead to cyclic errors or convergence to local extrema.

\noindent \textbf{Expectations:}
\begin{itemize}
    \item \textbf{Newton’s Method}: Quick convergence is possible with close initial guesses. However, the oscillatory nature may cause poor initial guesses to diverge or settle at local extrema.
    \item \textbf{Secant Method}: More robust without requiring derivatives, though it may face erratic convergence due to periodic behavior, especially if starting points are not well chosen.
\end{itemize}

\noindent \textbf{Predictions:}  
Both Newton’s and secant methods might exhibit inconsistent convergence depending on initial guesses due to the function’s periodic nature. Careful initial guesses and interval analysis are essential for reliable root-finding.

\subsection*{Overall Analysis}

While all methods have strengths, the examples illustrate potential weaknesses. The Lagrange and ideal gas law equations may present issues due to closely spaced or degenerate roots, while the sinusoidal motion equation could cause convergence challenges due to its periodic nature. These cases will help assess the methods’ limits and behavior in varied conditions.

\end{document}
