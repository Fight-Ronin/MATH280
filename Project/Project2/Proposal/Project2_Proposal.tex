\documentclass[12pt]{article}
\usepackage{amsmath, amsthm, amssymb, amsfonts}
\usepackage[margin=1in]{geometry}
\usepackage[backend=biber]{biblatex}
\bibliography{app_log.bib}

\newcounter{entry}
\newcommand\ALEntry[1]{\stepcounter{entry}\textbf{[\theentry]} \fullcite{#1}.\\[-0.2em]}

\title{\vspace{-2cm}Project 2 Proposal}
\author{Hanzhang Yin}

\begin{document}
\maketitle

\noindent In this project, we will test our algorithms using examples from physics and mathematics, including univariate polynomial equations, a system of polynomial equations, and a non-polynomial equation for testing Newton's, Bisection and the secant methods for root apprximations. Each example includes a brief explanation of its scientific background and the potential challenges it presents for numerical methods.

\subsection*{Univariate Polynomial Equations}

\textbf{Polynomial 1: Lagrange Points in Orbital Mechanics}

\noindent In celestial mechanics, Lagrange points are positions where a small body can maintain its position relative to two larger bodies due to gravitational forces. The \( L_1 \) Lagrange point is found by solving a quintic polynomial:

\[
    x^5 - (3 + \mu)x^4 + (3 + 2\mu)x^3 - \mu x^2 + 2\mu x - \mu = 0,
\]

\noindent where \( x \) is the normalized distance and \( \mu \) is the mass ratio of the two bodies. This example is chosen for its complexity and the potential for numerical issues when \( \mu \) approaches zero, leading to multiple or closely spaced roots.
\\
\textbf{Degeneration:} As \( \mu \rightarrow 0 \), the polynomial simplifies, which we will connect to the original equation by analyzing numerical stability and how root multiplicity affects convergence.
\\
\textbf{Polynomial 2: Ideal Gas Law For Real Gas}

\noindent The modified ideal gas law for real gases can be expressed as:

\[
    P V^3 - (P b + R T)V^2 + a V - a b = 0,
\]

\noindent where \( P \) is pressure, \( T \) is temperature, \( R \) is the gas constant, and \( a \), \( b \) are gas-specific constants. This cubic polynomial can exhibit different root structures under varying conditions, such as near the critical point where multiple real roots merge.
\\
\textbf{Degeneration:} Approaching the \textbf{critical point} might render multiple real roots converging, challenging numerical solvers with potential ill-conditioned equations.

\subsection*{System of Polynomial Equations}

\textbf{Example: Coupled Spring-Mass System}

\noindent In a physics-based spring-mass system with two masses connected by springs, the equations of motion can be represented as:

\[
    k_1 x_1 - k_2 (x_2 - x_1) = m_1 a_1,
\]
\[
    k_2 (x_2 - x_1) + k_3 x_2 = m_2 a_2,
\]

\noindent where \( k_1, k_2, \) and \( k_3 \) are spring constants, \( x_1 \) and \( x_2 \) are displacements, and \( a_1 \) and \( a_2 \) are accelerations. This example is straightforward and allow testing for potential singular behaviors when \( k_2 \) is small or approaches zero.

\subsection*{Non-Polynomial Equation}

\textbf{Example: Sinusoidal Motion in Spring Displacement}

\noindent The motion of a mass-spring system can be described by:

\[
    x(t) = 5 \sin(2t),
\]

\noindent or for finding specific points ($x = 3$ in this case):

\[
    5 \sin(2t) - 3 = 0.
\]

\noindent This non-polynomial equation will be used to test Newton’s method and the secant method. Solving for \( t \) involves periodic behavior, which can pose challenges for convergence due to the trigonometric nature of the function and the need for good initial guesses.

\subsection*{Anticipated Challenges and Degeneracies}
The Lagrange polynomial and the modified ideal gas law equation present cases with potential singularities and closely spaced roots, which can slow down the convergence. The non-polynomial trigonometric equation may experience convergence issues if the initial guesses are far from the actual roots due to the oscillatory behavior (For Newton's Method Especially).

\end{document} 
