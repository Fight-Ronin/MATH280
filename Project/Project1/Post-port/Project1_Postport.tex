\documentclass[12pt]{article}
\usepackage{amsmath, amsthm, amssymb, amsfonts}
\usepackage[margin=1in]{geometry}
\usepackage[backend=biber]{biblatex}
\bibliography{app_log.bib}

\newcounter{entry}
\newcommand\ALEntry[1]{\stepcounter{entry}\textbf{[\theentry]} \fullcite{#1}.\\[-0.2em]}

\title{\vspace{-2cm}Project 1 Post-port}
\author{Hanzhang Yin}


\begin{document}

\maketitle

\section*{Summary of Results}

This project aimed to distinguish the tonal styles of Zhu Shuzhen and Du Fu using Markov matrices and various computational methods. The test sequence was analyzed using two approaches: the log-likelihood method and a comparison of equilibrium vectors using cosine similarity and Euclidean distance.

\subsection*{Log-Likelihood Method}

The log-likelihood method correctly predicted Zhu Shuzhen as the author of the test tones. The total log-likelihoods were computed as follows:

\[
L_{Zhu} = 3 \log\left( \frac{35}{103} \right) + 5 \log\left( \frac{53}{156} \right) + 4 \log\left( \frac{61}{199} \right) + \ldots = \text{(summed result)}
\]
\[
L_{Du} = 3 \log\left( \frac{27}{80} \right) + \log\left( \frac{1}{3} \right) + 5 \log\left( \frac{88}{279} \right) + \ldots = \text{(summed result)}
\]

\noindent With a higher total log-likelihood for Zhu Shuzhen, this method accurately captured the tonal transitions characteristic of her style. 
\\
The reason this method worked well is that it directly uses the transition probabilities from the Markov matrix, which are based on observed tonal patterns in each poet's work. Even with a small test case, the log-likelihood method effectively quantifies how likely the transitions are for each poet, resulting in an accurate prediction. This approach is particularly robust when dealing with small datasets, as it inherently accounts for the probabilities of each transition in a cumulative manner.


\subsection*{Cosine Similarity and Euclidean Distance Methods}

An alternative method was employed, comparing the equilibrium vectors of the Markov matrices using cosine similarity and Euclidean distance. However, both methods incorrectly predicted Du Fu as the author. The detailed results were:

\begin{itemize}
    \item \textbf{Cosine Similarity:}
    \begin{itemize}
        \item With Zhu Shuzhen: 0.5729
        \item With Du Fu: 0.6121
    \end{itemize}
    \item \textbf{Euclidean Distance:}
    \begin{itemize}
        \item To Zhu Shuzhen: 0.8218
        \item To Du Fu: 0.7970
    \end{itemize}
\end{itemize}

\noindent Both methods favored Du Fu based on these metrics, which may be due to the following reasons:
\begin{itemize}
    \item \textbf{Small Test Case Size:} The limited size of the test sequence means that the calculated Markov matrix and equilibrium vector do not fully capture the general stylistic information. This can lead to inaccurate comparisons when using vector-based methods like cosine similarity and Euclidean distance.
    \item \textbf{Overfitting to Local Patterns:} The equilibrium vector derived from the small dataset may overemphasize specific transitions that are not representative of Zhu Shuzhen’s overall style, leading to closer matches with Du Fu's matrix.
    \item \textbf{Sensitivity of Similarity Measures:} Cosine similarity and Euclidean distance are sensitive to the direction and magnitude of the vectors. Given the sparse data, these measures may incorrectly favor Du Fu's equilibrium vector.
\end{itemize}

\subsection*{Baseline Infinity Norm Method}

A baseline infinity norm method was also applied to compare the differences between the equilibrium vectors of the test tones and the known styles of Zhu Shuzhen and Du Fu. The results were as follows:

\textbf{Baseline Infinity Norm Prediction for Zhu Shuzhen's test tones:}
\begin{itemize}
    \item Infinity Norm Difference with Zhu Shuzhen: 0.07692068
    \item Infinity Norm Difference with Du Fu: 0.09668192553969868
    \item Baseline Infinity Norm Prediction: Zhu Shuzhen
\end{itemize}

\textbf{Baseline Infinity Norm Prediction for Du Fu's test tones:}
\begin{itemize}
    \item Infinity Norm Difference with Zhu Shuzhen:
    \item Infinity Norm Difference with Du Fu: 
    \item Baseline Infinity Norm Prediction: 
\end{itemize}

\noindent The base line method provides correct prediction on Zhu Shuzhen's test tones

\subsection*{Conclusion}

The log-likelihood provided accurate results due to its probabilistic nature, directly utilizing the transition probabilities from the Markov matrices, making it more reliable even with small datasets.
The vector-norm method provided accurate results as well, my intuition on this behavior is that the simplicity of the calcuation process captures the generic patterns of the provided tones when it compare to test tones.
\\
In contrast, the cosine similarity, Euclidean distance, and baseline infinity norm methods all yielded incorrect predictions. Specifically, the cosine similarity values of 0.5729 with Zhu Shuzhen and 0.6121 with Du Fu, Euclidean distances of 0.8218 to Zhu Shuzhen and 0.7970 to Du Fu, and infinity norm differences of 0.2093 with Zhu Shuzhen and 0.1896 with Du Fu, highlight the difficulty in capturing Zhu Shuzhen's tonal patterns using these vector-based methods.
\\
Overall, the results demonstrate the effectiveness of the log-likelihood method in authorship attribution tasks, especially when data is limited. The vector-based methods may require larger datasets or further refinement to better account for stylistic nuances in tonal transitions.

\end{document} 