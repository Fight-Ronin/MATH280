\documentclass[12pt]{article}
\usepackage{amsmath, amsthm, amssymb, amsfonts}
\usepackage[margin=1in]{geometry}
\usepackage[backend=biber]{biblatex}
\bibliography{app_log.bib}

\newcounter{entry}
\newcommand\ALEntry[1]{\stepcounter{entry}\textbf{[\theentry]} \fullcite{#1}.\\[-0.2em]}

\title{\vspace{-2cm}Project 1 Summary}
\author{Hanzhang Yin}


\begin{document}
\maketitle

The Project focusing on using Markov models to analyze the tonal patterns in classical Chinese poetry, paticularly examining whether sequences of tones align with the strict metrical rules of 
(lv shī). These rules dictate specific tonal arragnements between \textit{even tone} and \textit{oblique tones} within each line of a poem, similar to the stressed and unstressed syllables arranged in metered English
poetry. The goal of this project is to construct a Markov matrix to capture the probabilities of transition between different tones, find the equilibrium distribution, and use it to gain 
insights into whether the observed tonal patterns match the expected ones.

The tonal pattern in (lv shī) have a direct analogy to metrical patterns in English poetry, such as iamic penameter, trochaic tetrameter, and other structured forms where syllable stress plays a defining role.
In classical Chinese poetry, each character (or syllable) is categorized intor either even (stable tone), or oblique (rising / falling tone). This binary cateogization resembles the \textit{stressed} and \textit{unstressed} syllables 
in English Prosody.

% For example, in English poetry, one of the most famous example is iambic pentameter, which consists of five pairs of alternating unstressed-stressed syllables (da-DUM, da-DUM, etc.). 
% This regular pattern creates a rhythmic flow, which poets can manipulate for emphasis, tension, or variation.
% Similarly, deviations from the expected even/oblique patterns in "lv shi" can indicate emphasis, emotion, or a deliberate break from tradition.

Markov matrices are tools used to model transitions between states, where each element \( M_{ij} \) represents the probability of moving from state \( i \) to state \( j \). 
In the context of English poetry meters, we can define states as \textit{stressed} (S) or \textit{unstressed} (U) syllables, or even larger metrical units like \textit{iambs} (U-S) and \textit{trochees} (S-U). 
For instance, an iambic pentameter line alternates between U and S, which can be represented by a simple Markov matrix: 

\[
    M = \begin{bmatrix} 0 & 1 \\ 1 & 0 \end{bmatrix}
\]
Indicating that U is always followed by S, and vice versa. By expanding the matrix to allow for deviations (e.g., occasional trochees or spondees), we can analyze the stylistic flexibility of a poem.
Furthermore, computing the \textit{equilibrium distribution} of the matrix reveals the long-term proportion of different metrical units, which helps identify dominant rhythmic patterns.

Lastly, here are some additional linguistic features that might be interesting using Markov methods to analyze under the hood:
\begin{itemize}
    \item Rhyme Schemes: Could applied to different poetry under different ``Dynasty'' to study how rhymes evolve and how certain poets deviate from common schemes.
    \item Sentence Length and Complexity: Study transitions between short and long sentences, or simple vs. complex clauses, especially in prose or in political speeches, to identify style...
\end{itemize}

\end{document} 
