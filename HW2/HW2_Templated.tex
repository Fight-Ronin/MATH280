\documentclass{article}
\usepackage[margin=1in]{geometry}
\usepackage{amsmath, amsthm, amssymb, amsfonts}
\usepackage{graphicx}
\usepackage{xcolor}
\usepackage{enumerate}
\newcounter{pnum}
\newcommand\problem[1]{\stepcounter{pnum}\begin{center}\fbox{\begin{minipage}{0.95\linewidth}
\textbf{\thepnum.} #1
\end{minipage}}\end{center}}
\newcommand\solution{\vspace{0.4em}\textbf{Solution.}}
\title{Homework 2 (power series, VES, error propagation)}
\author{Hanzhang Yin}
\begin{document}

\maketitle

\problem{Taylor's theorem and remainder.
\begin{enumerate}[\indent\indent a)]
    \item Write the degree \(5\) power series approximation for \(\sin x\) at \(x=\pi/4\). Also state the integral form of the remainder
    (Section 1.1, Theorem 5).
    \item Use your approximation to estimate \(\sin(0)\), \(\sin\pi/2\) and \(\sin 1\).
    item Use your remainder to bound the error on your estimates. What do you think about the bounds?
\end{enumerate}
}
\solution

\subsection*{(a)}
For $sinx$, the derivatives for constructing degree 5 power series approximation and error bound are:
\[ f'(x) = cosx, f''(x) = -sinx, f^{(3)}(x) = -cosx, f^{(4)}(x) = sinx, f^{(5)}(x) = cosx, f^{(6)}(x) = -sinx \]
The degree 5 Taylor series approximation for $\sin x$ around $x = \frac{\pi}{4}$ is:
\[ \sin(x) \approx \sin\left(\frac{\pi}{4}\right) + \cos\left(\frac{\pi}{4}\right)\left(x - \frac{\pi}{4}\right) - \frac{1}{2}\sin\left(\frac{\pi}{4}\right)\left(x - \frac{\pi}{4}\right)^2 - \frac{1}{6}\cos\left(\frac{\pi}{4}\right)\left(x - \frac{\pi}{4}\right)^3 \]
\[ + \frac{1}{24}\sin\left(\frac{\pi}{4}\right)\left(x - \frac{\pi}{4}\right)^4 + \frac{1}{120}\cos\left(\frac{\pi}{4}\right)\left(x - \frac{\pi}{4}\right)^5 + R_n(x) \]  
\\
Substituting the values for  $\sin\left(\frac{\pi}{4}\right)$ and $\cos\left(\frac{\pi}{4}\right)$ both as $\frac{\sqrt{2}}{2}$, we get:
\[ \sin(x) \approx \frac{\sqrt{2}}{2} + \frac{\sqrt{2}}{2}\left(x - \frac{\pi}{4}\right) - \frac{\sqrt{2}}{4}\left(x - \frac{\pi}{4}\right)^2 - \frac{\sqrt{2}}{12}\left(x - \frac{\pi}{4}\right)^3 + \frac{\sqrt{2}}{48}\left(x - \frac{\pi}{4}\right)^4 + \frac{\sqrt{2}}{240}\left(x - \frac{\pi}{4}\right)^5 \]
\\
The integral form of the remainder for the Taylor series of $\sin x$ around $x = \frac{\pi}{4}$ of the fifth degree by definition is:
\[ R_5(x) = \int_{\frac{\pi}{4}}^x \frac{f^{(6)}}{120} (x - t)^5 \ dt = -\int_{\frac{\pi}{4}}^x \frac{\sin(t)}{120} (x - t)^5 \ dt \]

\subsection*{(b)}
\[ \sin(0) \approx \frac{\sqrt{2}}{2} + \frac{\sqrt{2}}{2}\left(0 - \frac{\pi}{4}\right) - \frac{\sqrt{2}}{4}\left(0 - \frac{\pi}{4}\right)^2 - \frac{\sqrt{2}}{12}\left(0 - \frac{\pi}{4}\right)^3 + \frac{\sqrt{2}}{48}\left(0 - \frac{\pi}{4}\right)^4 + \frac{\sqrt{2}}{240}\left(0 - \frac{\pi}{4}\right)^5 = 0.000202341 \]\
\[ \sin(\frac{\pi}{2}) \approx \frac{\sqrt{2}}{2} + \frac{\sqrt{2}}{2}\left(\frac{\pi}{2} - \frac{\pi}{4}\right) - \frac{\sqrt{2}}{4}\left(\frac{\pi}{2} - \frac{\pi}{4}\right)^2 - \frac{\sqrt{2}}{12}\left(\frac{\pi}{2} - \frac{\pi}{4}\right)^3 + \frac{\sqrt{2}}{48}\left(\frac{\pi}{2} - \frac{\pi}{4}\right)^4 + \frac{\sqrt{2}}{240}\left(\frac{\pi}{2} - \frac{\pi}{4}\right)^5 = 1.000252 \]
\[ \sin(1) \approx \frac{\sqrt{2}}{2} + \frac{\sqrt{2}}{2}\left(1 - \frac{\pi}{4}\right) - \frac{\sqrt{2}}{4}\left(1 - \frac{\pi}{4}\right)^2 - \frac{\sqrt{2}}{12}\left(1 - \frac{\pi}{4}\right)^3 + \frac{\sqrt{2}}{48}\left(1 - \frac{\pi}{4}\right)^4 + \frac{\sqrt{2}}{240}\left(1 - \frac{\pi}{4}\right)^5 = 0.84147 \]

\subsection*{(c)}
To bound the error in our estimates, we consider the remainder term \( R_5(x) \), defined as:

\[
    |R_5(x)| = \int_{\frac{\pi}{4}}^x \frac{\sin(t)}{120} (x - t)^5 \ dt
\]
Note that for \( \sin t \), the range of $|\sin t| \leq 1$. We can simplify error bound to:

\[
    |R_5(x)| \leq \frac{1}{5!} \int_{\pi/4}^{x} (x - t)^5 \, dt = \frac{(\frac{\pi}{4} - x)^6}{6!}
\]
This bound is small for \( x \rightarrow \frac{\pi}{4} \). A small number raised to the sixth power is tiny, and dividing by \( 6! \) further reduces it. Thus, the Taylor series approximation is highly accurate with such small error bound.



\problem{Suppose we have measured \(Y=(2,1)\) with some error described by the covariance matrix \(\Sigma_Y = \begin{bmatrix}
    0.2 & 1\\
    1 & 0.3\\
\end{bmatrix}\). Use the error propagation equation to calculate or estimate \(\Sigma_{f(Y)}\) for each of the following:
\begin{enumerate}[\indent\indent a)]
    \item \(f(X) = \begin{bmatrix}
    3 & 4 \\ 1 & 2
    \end{bmatrix}X + \begin{bmatrix}
    1 \\ 1
    \end{bmatrix}\)
    \item \(f(X) = \begin{bmatrix}
    X_1 ^2 + 3X_1 X_2 - 5\\
    X_2 - X_1
    \end{bmatrix}\)
\end{enumerate}
}
\solution
\subsection*{(a)}
\[ f(X) = \begin{bmatrix} 3 & 4 \\ 1 & 2 \end{bmatrix} X + \begin{bmatrix} 1 \\ 1 \end{bmatrix} \]
\\
Given $Y = (1,3)$, and $\sum_y = \begin{bmatrix} 0.2 & 1 \\ 1 & 0.3 \end{bmatrix}$.
\\
For linear function of the form $f(X) = AX + b$, the Jacobian $J_f$ is simply $A$, hence:
\[ \sum_{f(y)} = \begin{bmatrix} 3 & 4 \\ 1 & 2 \end{bmatrix} \begin{bmatrix} 0.2 & 1 \\ 1 & 0.3 \end{bmatrix} \begin{bmatrix} 3 & 1 \\ 4 & 2 \end{bmatrix} \]
\[ = \begin{bmatrix} 4.6 & 4.2 \\ 2.2 & 1.6 \end{bmatrix} \begin{bmatrix} 3 & 1 \\ 4 & 2 \end{bmatrix} = \begin{bmatrix} 30.6 & 13 \\ 13 & 5.4 \end{bmatrix} \]

\subsection*{(b)}
\[ f(X) = \begin{bmatrix} X_1^2 + 3X_1X_2 - 5 \\ X_2 - X_1 \end{bmatrix} \]
\\
Given $Y = (2,1)$, and $\sum_y = \begin{bmatrix} 0.2 & 1 \\ 1 & 0.3 \end{bmatrix}$.
\\
For Jacobian Matrix calculation, here is the requried partial derivatives:
\[ \frac{\partial}{\partial X_1} (X_1^2 + 3X_1X_2 - 5) = 2X_1 + 3X_2 \]
\[ \frac{\partial}{\partial X_2} (X_1^2 + 3X_1X_2 - 5) = 3X_1 \]
\[ \frac{\partial}{\partial X_1} (X_2 - X_1) = -1 \]
\[ \frac{\partial}{\partial X_2} (X_2 - X_1) = 1 \]
\\
At $Y = (2,1)$, we have:
\[ J_f= \begin{bmatrix} 2(2) + 3(1) & 3(2) \\ -1 & 1 \end{bmatrix} = \begin{bmatrix} 7 & 6 \\ -1 & 1 \end{bmatrix} \]
\\
Hence,
\[ \sum_{f(y)} = \begin{bmatrix} 7 & 6 \\ -1 & 1 \end{bmatrix} \begin{bmatrix} 0.2 & 1 \\ 1 & 0.3 \end{bmatrix} \begin{bmatrix} 7 & -1 \\ 6 & 1 \end{bmatrix} \]
\[ = \begin{bmatrix} 7.4 & 8.8 \\ 0.8 & -0.7 \end{bmatrix} \begin{bmatrix} 7 & -1 \\ 6 & 1 \end{bmatrix} = \begin{bmatrix} 104.6 & 1.4 \\ 1.4 & -1.5 \end{bmatrix} \]

\problem{Use the error propagation equation to give general expressions for \(\Sigma_{f(x,y)}\) when \(x\) and \(y\) are uncorrelated, meaning
\(\Sigma_{x,y} = \begin{bmatrix}
\sigma_x^2 & 0 \\0&\sigma_y^2
\end{bmatrix}\), in terms of \(X,Y,\sigma_x,\sigma_y\).
\begin{enumerate}[\indent\indent a)]
\item \(f(X,Y) = X+Y\).
\item \(f(X,Y) = XY\).
\item \(f(X,Y) = X/Y\).
\end{enumerate}
}
\solution
\subsection*{(a)}
\[ f(X, Y) = X + Y \]
\\
The Jacobian matrix of $f$ w.r.t. $X$ and $Y$ is:
\[ J_f = [1, 1] \]
\\
Thus, covariance matrix $\sum_{f(x,y)}$ is:
\[ \sum_{f(x,y)} =   \begin{bmatrix} 1 & 1 \end{bmatrix} \begin{bmatrix} \sigma^2_X & 0 \\ 0 & \sigma^2_Y \end{bmatrix} \begin{bmatrix} 1 \\ 1 \end{bmatrix} = \sigma^2_X + \sigma^2_Y \]

\subsection*{(b)}
\[ f(X, Y) = XY \]
\\
The Jacobian matrix of $f$ w.r.t. $X$ and $Y$ is:
\[ J_f = [Y, X] \]
\\
Thus, covariance matrix $\sum_{f(x,y)}$ is:
\[ \sum_{f(x,y)} =   \begin{bmatrix} Y & X \end{bmatrix} \begin{bmatrix} \sigma^2_X & 0 \\ 0 & \sigma^2_Y \end{bmatrix} \begin{bmatrix} Y \\ X \end{bmatrix} = Y^2 \sigma^2_X + X^2 \sigma^2_Y \]

\subsection*{(c)}
\[ f(X, Y) = \frac{X}{Y} \]
\\
The Jacobian matrix of $f$ w.r.t. $X$ and $Y$ is:
\[ J_f = [\frac{1}{Y}, -\frac{X}{Y^2}] \]
\\
Thus, covariance matrix $\sum_{f(x,y)}$ is:
\[ \sum_{f(x,y)} = \begin{bmatrix} \frac{1}{Y} & -\frac{X}{Y^2} \end{bmatrix} \begin{bmatrix} \sigma^2_X & 0 \\ 0 & \sigma^2_Y \end{bmatrix} \begin{bmatrix} \frac{1}{Y} \\ -\frac{X}{Y^2} \end{bmatrix} = \frac{\sigma^2_X}{Y^2} + \frac{X^2 \sigma^2_Y}{Y^4} \]

\problem{(bonus) The virial equation of state is for single-component gases, not mixtures. For an ideal mixture of ideal gases, one can break
up the ideal gas law using partial pressures:
\[P_i V = n_i RT\]
where \(n_i\) is the number of molecules of the \(i\)th constituent and \(P_i\) is the partial pressure of the \(i\)th consitutent (its molar
proportion times total pressure). These sum to the ideal gas law.
For example, if there are two components,
\[P_1 V = n_1 RT\]
\[P_2 V = n_2 RT\]
and \(P_1 + P_2 = P\), the total pressure, and \(n_1 + n_2 = n\), the total number of molecules.
In the non-ideal case, one might be inclined to introduce compressibility factors \(Z_{i,mix}\) for each gas (depending on the mixture),
\[P_i V = n_i RT Z_{i,mix}.\]
\((*)\) Propose a generalization of the virial equation of state to a two-component mixture. In other words, a power series expression for the
\(Z_1, Z_2\) with physical interpretations of their coefficients.
\((**)\) Assuming the gases are individually ideal, how could you simplify your equation of state?}
\solution
\\
\textbf{Virial Equation for a Single Component Gas: }
\\
The virial equation of state for a single component gas expands the ideal gas law to account for interactions between molecules. It is expressed as:
\[ PV = nRT \left(1 + B_2(T) \frac{n}{V} + B_3(T) \left(\frac{n}{V}\right)^2 + \cdots \right)
\]
\\
where \( B_2(T), B_3(T), \ldots \) are the virial coefficients that depend on temperature and represent interactions among molecules.
\\
\textbf{Extension to Two-Component Mixtures: }
\\
For a mixture of gases, the virial equation can be generalized as:
\[ PV = nRT \left(1 + \left(B_{11} \frac{n_1}{V} + B_{22} \frac{n_2}{V} + 2B_{12} \frac{n_1n_2}{V^2} \right) + \cdots \right) \]
\\
Here, \( B_{11}(T) \), \( B_{22}(T) \) are virial coefficients correspond to the interactions within the same type of molecules; \( B_{12}(T) \) is the virial coefficient that considers cross-interactions between two gas species.
\\
Note that \( x_1 \) and \( x_2 \) are the mole fractions of gas 1 and gas 2, and $n = n_1 + n_2$ is the total number of moles in the mixture.
\\
\textbf{Simplification Under Ideal Conditions: }\
\\
If both gases are ideal:
\begin{itemize}
    \item Interactions between molecules in their own gas species are negligible. Only cross-interactions
    between different gas species are left.
    \item This implies \( B_{11} = B_{22} = \ldots = 0 \).
\end{itemize}
\\
Following the same definition previously, the virial equation reduces to:
\[ PV = nRT \left( 1 + 2B_{12} \frac{n_1n_2}{V^2} + O(n_1, n_2) \right) \]
NOTE: $O(n_1, n_2)$: Higher order polynomial terms.
\\
For simplicity, since higher order terms from the generalized viral equation does not contribute significant impact to the calculation, it is 
sufficient to consider the equation as:
\[  PV \approx nRT \left( 1 + 2B_{12} \frac{n_1n_2}{V^2} \right)\]

\end{document}