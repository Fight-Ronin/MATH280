\documentclass{article}
\usepackage{graphicx} % Required for inserting images
\usepackage[utf8]{inputenc}
\usepackage{amsmath}
\usepackage{graphicx}
\usepackage{tikz}
\usepackage{array}
\usepackage{amssymb}
\newcommand*{\twoheadrightarrowtail}{\mathrel{\rightarrowtail\kern-1.9ex\twoheadrightarrow}}
% Alternative which doesn't look as good using the normal size, but might work better with smaller sizes too:
%\newcommand*{\twoheadrightarrowtail}{\mathrel{\rlap{$\rightarrowtail$}\twoheadrightarrow}}
\usepackage{amssymb}
\usepackage{amsthm}
\usepackage{multirow}
\usepackage{dcolumn}
\newcolumntype{2}{D{.}{}{2.0}}

\title{MATH 280 HW2}
\author{Hanzhang Yin}
\date{Sep/5/2024}

\begin{document}

\maketitle

\section*{Question 1}

\subsection*{(a)}
The degree 5 Taylor series approximation for $\sin x$ around $x = \frac{\pi}{4}$ is:
\[ \sin(x) \approx \sin\left(\frac{\pi}{4}\right) + \cos\left(\frac{\pi}{4}\right)\left(x - \frac{\pi}{4}\right) - \frac{1}{2}\sin\left(\frac{\pi}{4}\right)\left(x - \frac{\pi}{4}\right)^2 - \frac{1}{6}\cos\left(\frac{\pi}{4}\right)\left(x - \frac{\pi}{4}\right)^3 + \frac{1}{24}\sin\left(\frac{\pi}{4}\right)\left(x - \frac{\pi}{4}\right)^4 \]  
\\
Substituting the values for  $\sin\left(\frac{\pi}{4}\right)$ and $\cos\left(\frac{\pi}{4}\right)$ both as $\frac{\sqrt{2}}{2}$, we get:
\[ \sin(x) \approx \frac{\sqrt{2}}{2} + \frac{\sqrt{2}}{2}\left(x - \frac{\pi}{4}\right) - \frac{\sqrt{2}}{4}\left(x - \frac{\pi}{4}\right)^2 - \frac{\sqrt{2}}{12}\left(x - \frac{\pi}{4}\right)^3 + \frac{\sqrt{2}}{48}\left(x - \frac{\pi}{4}\right)^4 \]
\\
The integral form of the remainder for the Taylor series of $\sin x$ around $x = \frac{\pi}{4}$ and truncated after the fifth degree is:
\[ R_5(x) = \int_{\frac{\pi}{4}}^x \frac{\sin(t)}{120} (x - t)^5 \, dt \]

\subsection*{(b)}
\[ \sin(0) \approx \frac{\sqrt{2}}{2} + \frac{\sqrt{2}}{2}\left(0 - \frac{\pi}{4}\right) - \frac{\sqrt{2}}{4}\left(0 - \frac{\pi}{4}\right)^2 - \frac{\sqrt{2}}{12}\left(0 - \frac{\pi}{4}\right)^3 + \frac{\sqrt{2}}{48}\left(0 - \frac{\pi}{4}\right)^4 = 0.001963321 \]\
\[ \sin(\frac{\pi}{2}) \approx \frac{\sqrt{2}}{2} + \frac{\sqrt{2}}{2}\left(\frac{\pi}{2} - \frac{\pi}{4}\right) - \frac{\sqrt{2}}{4}\left(\frac{\pi}{2} - \frac{\pi}{4}\right)^2 - \frac{\sqrt{2}}{12}\left(\frac{\pi}{2} - \frac{\pi}{4}\right)^3 + \frac{\sqrt{2}}{48}\left(\frac{\pi}{2} - \frac{\pi}{4}\right)^4 = 0.9984927\]
\[ \sin(1) \approx \frac{\sqrt{2}}{2} + \frac{\sqrt{2}}{2}\left(1 - \frac{\pi}{4}\right) - \frac{\sqrt{2}}{4}\left(1 - \frac{\pi}{4}\right)^2 - \frac{\sqrt{2}}{12}\left(1 - \frac{\pi}{4}\right)^3 + \frac{\sqrt{2}}{48}\left(1 - \frac{\pi}{4}\right)^4 = 0.84146840 \]

\subsection*{(c)}
\[ R_5(0) = \int_{\frac{\pi}{4}}^{0} \frac{\sin(t)}{120} (0 - t)^5 \, dt = 0.00020235 \]
\[ R_5(\frac{\pi}{2}) = \int_{\frac{\pi}{4}}^{\frac{\pi}{2}} \frac{\sin(t)}{120} (\frac{\pi}{2} - t)^5 \, dt = 0.000243632 \]
\[ R_5(1) = \int_{\frac{\pi}{4}}^{1} \frac{\sin(t)}{120} (1 - t)^5 \, dt = 9.879 \times 10^{-8} \]
\\
These values indicate extremely small errors, showcasing the high accuracy of the Taylor series approximation when truncated at the fifth degree.

\begin{itemize}
    \item \textbf{At \(x = 0\)}: The error bound of \(0.00020235\) indicates that the approximation is nearly exact.
    \item \textbf{At \(x = \frac{\pi}{2}\)}: An error bound of \(0.00024362\) suggests that the approximation is extremely close to the actual value \(1\).
    \item \textbf{At \(x = 1\)}: The error bound of \(9.879 \times 10^{-8}\) indicates the approximation is indistinguishable from the true value \(\sin(1)\).
\end{itemize}
\\
These bounds confirm that the fifth-order Taylor series provides an excellent approximation for \(\sin x\) around \(x = \frac{\pi}{4}\), particularly near this point and at commonly evaluated points within the function's periodic range. 

\section*{Question 2}

\subsection*{(a)}
\[ f(X) = \begin{bmatrix} 3 & 4 \\ 1 & 2 \end{bmatrix} X + \begin{bmatrix} 1 \\ 1 \end{bmatrix} \]
\\
Given $Y = (1,3)$, and $\sum_y = \begin{bmatrix} 0.2 & 1 \\ 1 & 0.3 \end{bmatrix}$.
\\
For linear function of the form $f(X) = AX + b$, the Jacobian $J_f$ is simply $A$, hence:
\[ \sum_{f(y)} = \begin{bmatrix} 3 & 4 \\ 1 & 2 \end{bmatrix} \begin{bmatrix} 0.2 & 1 \\ 1 & 0.3 \end{bmatrix} \begin{bmatrix} 3 & 1 \\ 4 & 2 \end{bmatrix} \]
\[ = \begin{bmatrix} 4.6 & 4.2 \\ 2.2 & 1.6 \end{bmatrix} \begin{bmatrix} 3 & 1 \\ 4 & 2 \end{bmatrix} = \begin{bmatrix} 30.6 & 13 \\ 13 & 5.4 \end{bmatrix} \]

\subsection*{(b)}
\[ f(X) = \begin{bmatrix} X_1^2 + 3X_1X_2 - 5 \\ X_2 - X_1 \end{bmatrix} \]
\\
Given $Y = (2,1)$, and $\sum_y = \begin{bmatrix} 0.2 & 1 \\ 1 & 0.3 \end{bmatrix}$.
\\
For Jacobian Matrix calculation, here is the requried partial derivatives:
\[ \frac{\partial}{\partial X_1} (X_1^2 + 3X_1X_2 - 5) = 2X_1 + 3X_2 \]
\[ \frac{\partial}{\partial X_2} (X_1^2 + 3X_1X_2 - 5) = 3X_1 \]
\[ \frac{\partial}{\partial X_1} (X_2 - X_1) = -1 \]
\[ \frac{\partial}{\partial X_2} (X_2 - X_1) = 1 \]
\\
At $Y = (2,1)$, we have:
\[ J_f= \begin{bmatrix} 2(2) + 3(1) & 3(2) \\ -1 & 1 \end{bmatrix} = \begin{bmatrix} 7 & 6 \\ -1 & 1 \end{bmatrix} \]
\\
Hence,
\[ \sum_{f(y)} = \begin{bmatrix} 7 & 6 \\ -1 & 1 \end{bmatrix} \begin{bmatrix} 0.2 & 1 \\ 1 & 0.3 \end{bmatrix} \begin{bmatrix} 7 & -1 \\ 6 & 1 \end{bmatrix} \]
\[ = \begin{bmatrix} 7.4 & 8.8 \\ 0.8 & -0.7 \end{bmatrix} \begin{bmatrix} 7 & -1 \\ 6 & 1 \end{bmatrix} = \begin{bmatrix} 194.6 & 1.4 \\ 1.4 & -1.5 \end{bmatrix} \]

\section*{Question 3}
Given, $\sum_{x,y} = \begin{bmatrix} \sigma^2_X & 0 \\ 0 & \sigma^2_Y \end{bmatrix}$.
\subsection*{(a)}
\[ f(X, Y) = X + Y \]
\\
The Jacobian matrix of $f$ w.r.t. $X$ and $Y$ is:
\[ J_f = [1, 1] \]
\\
Thus, covariance matrix $\sum_{f(x,y)}$ is:
\[ \sum_{f(x,y)} =   \begin{bmatrix} 1 & 1 \end{bmatrix} \begin{bmatrix} \sigma^2_X & 0 \\ 0 & \sigma^2_Y \end{bmatrix} \begin{bmatrix} 1 \\ 1 \end{bmatrix} = \sigma^2_X + \sigma^2_Y \]

\subsection*{(b)}
\[ f(X, Y) = XY \]
\\
The Jacobian matrix of $f$ w.r.t. $X$ and $Y$ is:
\[ J_f = [Y, X] \]
\\
Thus, covariance matrix $\sum_{f(x,y)}$ is:
\[ \sum_{f(x,y)} =   \begin{bmatrix} Y & X \end{bmatrix} \begin{bmatrix} \sigma^2_X & 0 \\ 0 & \sigma^2_Y \end{bmatrix} \begin{bmatrix} Y \\ X \end{bmatrix} = Y^2 \sigma^2_X + X^2 \sigma^2_Y \]

\subsection*{(c)}
\[ f(X, Y) = \frac{X}{Y} \]
\\
The Jacobian matrix of $f$ w.r.t. $X$ and $Y$ is:
\[ J_f = [\frac{1}{Y}, -\frac{X}{Y^2}] \]
\\
Thus, covariance matrix $\sum_{f(x,y)}$ is:
\[ \sum_{f(x,y)} = \begin{bmatrix} \frac{1}{Y} & -\frac{X}{Y^2} \end{bmatrix} \begin{bmatrix} \sigma^2_X & 0 \\ 0 & \sigma^2_Y \end{bmatrix} \begin{bmatrix} \frac{1}{Y} \\ -\frac{X}{Y^2} \end{bmatrix} = \frac{\sigma^2_X}{Y^2} + \frac{X^2 \sigma^2_Y}{Y^4} \]

\section*{Bonus Question}
Given the viral equation in ideal gas:
\[ P_i V = n_i R T \]
\\
In non-ideal gas case, we might introduce compressibility factors $Z_{i, mix}$ for each gas:
\[ P_i V = n_i R T Z_{i, mix} \]

\subsection*{Propose a generalization of the virial equation of state to a two-component mixture}

\textbf{Virial Equation for a Single Component Gas: }
\\
The virial equation of state for a single component gas expands the ideal gas law to account for interactions between molecules. It is expressed as:
\[ P = \frac{nRT}{V} \left(1 + \frac{B_2(T)}{V} + \frac{B_3(T)}{V^2} + \cdots \right) \]
\\
where \( B_2(T), B_3(T), \ldots \) are the virial coefficients that depend on temperature and represent interactions among molecules.
\\
\textbf{Extension to Two-Component Mixtures: }
\\
For a mixture of gases, the virial equation can be generalized as:
\[ P = \frac{(n_1+n_2)RT}{V} \left(1 + \frac{B_{12}(T)}{V} + \frac{C_{123}(T)}{V^2} + \cdots \right) \]
\\
Here, \( B_{12}(T) \) is the second virial coefficient that considers cross-interactions between different gas species and is approximated as:
\[ B_{12}(T) = x_1^2 B_{11}(T) + 2x_1x_2 B_{12}(T) + x_2^2 B_{22}(T) \]
\\
where \( x_1 \) and \( x_2 \) are the mole fractions of gas 1 and gas 2, respectively.

\subsection*{Simplification Under Ideal Conditions}
If both gases are ideal:
\begin{itemize}
    \item Interactions between molecules and between different species are negligible.
    \item This implies \( B_2 = B_3 = \ldots = 0 \).
\end{itemize}

Thus, the virial equation reduces to:
\[ P = \frac{nRT}{V} \]
where \( n = n_1 + n_2 \), indicating that the pressure is determined purely by the ideal gas contributions of both components, unaffected by intermolecular interactions.

\end{document}
